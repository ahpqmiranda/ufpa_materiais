\section{Introdução}\label{Intro}

A história da humanidade se entrelaça com a dos tratamentos térmicos, ao longo de milhares de anos atrás, diversos povos, chineses, egípcios e indianos já conheciam a técnica de endurecimento de metais por meio de aquecimento e resfriamento, tendo seu expoente histórico na idade do ferro. O processo era utilizado principalmente para fabricação de armas, ferramentas e utensílios domésticos.

Na Idade Média, os europeus desenvolveram técnicas avançadas de tratamento térmico para a fabricação de armas. Nesta época, o processo de têmpera foi aperfeiçoado e se tornou amplamente utilizado para produzir espadas, lanças, armaduras e outras armas de guerra. O processo também foi utilizado para fabricar ferramentas e instrumentos de corte para diversas aplicações.

O processo de tempera é um tratamento térmico utilizado para melhorar as propriedades mecânicas de materiais, tornando-os mais duros e resistentes. A necessidade de usar o processo de tempera vem da necessidade de melhorar as propriedades mecânicas de um material, tornando-o mais resistente e durável. Isso é particularmente importante para materiais usados em aplicações que sofrem desgaste ou estresse mecânico intenso.

A aplicação do processo de tempera pode ser realizada em uma ampla variedade de materiais, incluindo aços carbono, aços liga, ligas de alumínio e titânio, entre outros. O processo é particularmente eficaz para aços carbono de alta resistência, como o aço SAE 1045.

No caso de um parafuso de aço SAE 1045 aquecido até 900ºC, o processo de tempera seria aplicado para produzir uma estrutura martensítica no material. Isso tornaria o parafuso extremamente duro e resistente, adequado para aplicações em que a resistência à tração e a dureza são importantes. No entanto, é importante ter em mente que a alta dureza pode tornar o parafuso mais propenso a trincas e rupturas em situações de alto impacto ou vibração, portanto, é importante escolher o tratamento térmico correto para atender às propriedades necessárias para a aplicação específica.