\documentclass[a4paper, 12pt]{article}
\usepackage[left=3cm, top=3cm, right=2cm, bottom=2cm]{geometry}
%pacote que define as margens
\usepackage[utf8]{inputenc} %pacote que normaliza a acentuação
\usepackage[brazil]{babel} %pacote que define o idioma
\usepackage{graphicx} %pacote que permite a inserção de imagens
\pagestyle{myheadings} %pacote que define o cabeçalho com o número de página no canto superior direito
\usepackage{setspace} %pacote que permite especificar o espaçamento entre linhas no documento
\usepackage{indentfirst} %pacote que faz indentação na primeira linha do parágrafo
\setlength{\parindent}{1cm} %pacote que define o tamanho da indentação
\usepackage{float} %Pacote que permite tabelas flutuarem em qualquer posição
\usepackage[alf,abnt-repeated-title-omit=yes,abnt-emphasize=bf,abnt-etal-list=0]{abntex2cite} %define os parâmetros necessários para as referências e citações de acordo com a ABNT
\usepackage{amsthm}
\usepackage{amsmath}
\usepackage{amsfonts}
\usepackage{subcaption} % Pacote que permite que imagens possam ser inseridas em composições de imagens lado a lado com o \subfigure
\usepackage{amssymb} %4 Pacotes responsáveis pela formatação matemática
\usepackage{rotating} %permite que textos, figuras e tabelas possam ser rotacionados

\begin{document}
	
	%! Author = alanmiranda
%! Date = 14/11/2022
\thispagestyle{empty}
    \begin{center}
        \parbox{3cm}{\includegraphics[scale=1]{pictures/logo_ufpa}}\\
        \vspace{1cm}
        \Large \uppercase{Universidade Federal do Pará}\\
        \Large \uppercase{Instituto de Tecnologia - ITEC}\\
        \vspace{3cm}
        \Large \uppercase{Faculdade de Engenharia Mecânica - FEM}\\
        \Large \uppercase{Laboratório de Química Analítica Quantitativa - LQAQ}\\
        \vspace{3cm}
        \Large \textbf{\uppercase {Relatório de Prática 7: Utilização de Indicadores Ácidos e Bases}} \\
        \Large \textbf{\uppercase {PROF. DR. Carlos Antônio Neves}} \\
        \vspace{3cm}
        \Large \uppercase {Alan Henrique Pereira Miranda - 202102140072}\\
        \Large \uppercase {Gabriel Cruz de Oliveira - 202102140055}\\
        \Large \uppercase {Paloma Gama da Silva - 202102140029}\\
        \Large \uppercase {Silvio Farias Leal - 202102140035}\\
        \vspace{1cm}
        \Large {Belém-PA \\ 2022}

    \end{center}

	
	\section{Introdução}\label{Intro}

A história da humanidade se entrelaça com a dos tratamentos térmicos, ao longo de milhares de anos atrás, diversos povos, chineses, egípcios e indianos já conheciam a técnica de endurecimento de metais por meio de aquecimento e resfriamento, tendo seu expoente histórico na idade do ferro. O processo era utilizado principalmente para fabricação de armas, ferramentas e utensílios domésticos.

Na Idade Média, os europeus desenvolveram técnicas avançadas de tratamento térmico para a fabricação de armas. Nesta época, o processo de têmpera foi aperfeiçoado e se tornou amplamente utilizado para produzir espadas, lanças, armaduras e outras armas de guerra. O processo também foi utilizado para fabricar ferramentas e instrumentos de corte para diversas aplicações.

O processo de tempera é um tratamento térmico utilizado para melhorar as propriedades mecânicas de materiais, tornando-os mais duros e resistentes. A necessidade de usar o processo de tempera vem da necessidade de melhorar as propriedades mecânicas de um material, tornando-o mais resistente e durável. Isso é particularmente importante para materiais usados em aplicações que sofrem desgaste ou estresse mecânico intenso.

A aplicação do processo de tempera pode ser realizada em uma ampla variedade de materiais, incluindo aços carbono, aços liga, ligas de alumínio e titânio, entre outros. O processo é particularmente eficaz para aços carbono de alta resistência, como o aço SAE 1045.

No caso de um parafuso de aço SAE 1045 aquecido até 900ºC, o processo de tempera seria aplicado para produzir uma estrutura martensítica no material. Isso tornaria o parafuso extremamente duro e resistente, adequado para aplicações em que a resistência à tração e a dureza são importantes. No entanto, é importante ter em mente que a alta dureza pode tornar o parafuso mais propenso a trincas e rupturas em situações de alto impacto ou vibração, portanto, é importante escolher o tratamento térmico correto para atender às propriedades necessárias para a aplicação específica.
	
	%! Author = alanmiranda
%! Date = 14/11/2022
\begin{itemize}
        \item Identificar a presença de ácidos e bases em soluções, utilizando indicadores ácidos e bases.
        \item Determinar a cor de transição de cada indicador utilizado.
    \end{itemize}
    \subsection{Objetivos específicos}\label{sec:objetivos_especificos}
    \indent Verificar o comportamento de cada indicador ácido e base, quando exposto a soluções ácidas e básicas.
	
	
	\addcontentsline{toc}{section}{Referências} %adiona a seção das referências no sumário
	\bibliographystyle{abntex2-alf}
	\bibliography{referencias}
	
\end{document}
