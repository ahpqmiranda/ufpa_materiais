\newpage


\section{Metodologia}
Para a realização do experimento, foi utilizado um dinamômetro, um paquímetro, uma mola helicoidal, um suporte para a mola, um suporte para o dinamômetro e um suporte para o paquímetro.
O experimento foi realizado em 4 etapas, sendo elas:

\begin{enumerate}
    \item Medição do diâmetro do fio.
    \item Medição do diâmetro médio da mola.
    \item Medição do número de espiras.
    \item Medição da relação entre a força aplicada e a deformação da mola.
\end{enumerate}

\subsection{Medições de Parâmetros Geométricos}
O diâmetro do fio foi medido utilizando um paquímetro, com uma resolução de 0,05 mm, e foi medido em 3 pontos diferentes, para garantir a precisão da medição.
Também foram medidas as dimensões da mola, como o diâmetro externo, o diâmetro interno e a altura da mola, para que fosse possível calcular o diâmetro médio da mola.
Por fim, foi contado o número de espiras da mola, chegando na seguinte relação:

% Please add the following required packages to your document preamble:
% \usepackage{graphicx}
\begin{center}
    \begin{table}[h]
        \centering
        \begin{tabular}{|c|c|c|c|}
            \rowcolor[HTML]{DAE8FC}
            \hline
            index & d\_ext (mm) & d\_int (mm) & d\_fio (mm) \\ \hline
            t1   & 44.4        & 37.8        & 2.51        \\ \hline
            t2   & 44.33       & 37.87       & 2.53        \\ \hline
            t3   & 44.35       & 37.82       & 2.49        \\ \hline
            t4   & 44.42       & 37.79       & 2.51        \\ \hline
            t5   & 44.4        & 37.77       & 2.5         \\ \hline
        \end{tabular}
        \caption{Dados obtidos no experimento.}
        \label{tab:dados_adquiridos}
        \textbf{Fonte:} Elaborado pelo autor (2023).
    \end{table}
\end{center}
