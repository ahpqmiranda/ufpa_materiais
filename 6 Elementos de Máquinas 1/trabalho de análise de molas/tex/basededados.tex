\section{Fundamentação Teórica}
O estudo do comportamento de molas é uma área importante para diversas áreas relacionadas a física e a engenharia, pois,
diversos materiais demonstram propriedade elástica em seu comportamento natural, como por exemplo, aço, borracha, madeira, etc.

Todo material que sofre deformação quando submetido a uma força externa e retorna a sua forma original quando a força é removida é chamado de material elástico.
A deformação elástica é reversível, ou seja, quando a força é removida, o material retorna a sua forma original. para além disso, consideramos que a deformação é
plástica ou elastoplástica. Quando um determinado material atinge tal condição, ele não retorna a sua forma original, mesmo após a remoção da força externa.

Tal característica é muito estudada dentro da engenharia, pois, a partir do conhecimento do comportamento de molas, é possível criar mecanismos que utilizem
tal propriedade em benefício do projeto, como por exemplo, amortecedores, sistemas de suspensão, sistemas de frenagem, etc.

A aproximação teórica do comportamento das molas é feita através de análises matemáticas, sendo a mais conhecida, a Lei de Hooke, que relaciona a força aplicada
em uma mola com a deformação sofrida por ela. Sendo ela uma aproximação linear do comportamento de molas, ou seja, ela só é válida para pequenas deformações.

\subsection{Lei de Hooke}
A Lei de Hooke é uma aproximação linear do comportamento de molas, ou seja, ela só é válida para pequenas deformações. A Lei de Hooke é dada pela seguinte equação:

\begin{equation}
    F = k \cdot \Delta L
\end{equation}

Onde:
\begin{enumerate}
    \item k : constante elástica da mola (N/m)
    \item $\Delta$L : deformação da mola (m)
    \item F : força aplicada na mola (N)
\end{enumerate}

Quando tratamos de um corpo rígido, podemos calcular a constante elástica, como o quociente entre a força aplicada e a deformação da mola, ou seja:

\begin{equation}
    k = \frac{F}{\Delta L}
\end{equation}

Mas esta abordagem considera apenas fatores geométricos, como o comprimento da mola e a deformação sofrida por ela, não dizendo muito sobre a disposição geométrica e como isso afeta o comportamento da mola, como no caso de molas helicoidais.

\subsection{Molas Helicoidais}
As molas helicoidais são as mais utilizadas em sistemas mecânicos, pois, são fáceis de fabricar e possuem uma grande variedade de aplicações.
A mola helicoidal é um dispositivo mecânico que possui a capacidade de armazenar energia mecânica, quando submetida a uma força externa, e liberá-la quando a força é removida, porém, sua geometria, composição e características construtivas buscam otimizar tal comportamento elástico, adequando-se a sua aplicação comercial/industrial.

\begin{figure}[h]
	\centering
	\includegraphics[width=0.7\linewidth]{"pics/geometria da mola"}
	\caption{Esquemático: Componentes Geométricos de uma Mola}
	\label{fig:geometria-da-mola}
\end{figure}



A mola helicoidal é composta por um fio de seção transversal circular, enrolado em forma de hélice, com um diâmetro de espira, diâmetro do fio, e com as extremidades retificadas, para que possam ser fixadas em um sistema mecânico.
Tais propriedades são levadas em consideração para estimar com precisão, o valor da constante elástica da mola.


A constante elástica de uma mola é dada pela seguinte equação:

\begin{equation}
    k = \frac{G \cdot d^4}{8 \cdot D^3 \cdot n}
\end{equation}

Onde:

\begin{enumerate}
    \item G : módulo de elasticidade transversal do material (Pa)
    \item d : diâmetro do fio (m)
    \item D : diâmetro médio da mola (m)
    \item n : número de espiras
    \item k : constante elástica da mola (N/m)
\end{enumerate}



