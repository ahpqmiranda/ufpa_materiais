%! Author = ahpqmiranda
%! Date = 11/20/23

% Preamble
\documentclass[11pt]{article}
\usepackage[left=3cm, top=3cm, right=2cm, bottom=2.0cm]{geometry} %pacote que define as margens do documento
% Packages
\usepackage[utf8]{inputenc} %pacote que normaliza a acentuação
\usepackage[brazil]{babel} %pacote que define o idioma
\usepackage{graphicx} %pacote que permite a inserção de imagens
\pagestyle{myheadings} %pacote que define o cabeçalho com o número de página no canto superior direito
\usepackage{setspace} %pacote que permite especificar o espaçamento entre linhas no documento
\usepackage{indentfirst} %pacote que faz indentação na primeira linha do parágrafo
\setlength{\parindent}{1cm} %pacote que define o tamanho da indentação
\usepackage{float} %Pacote que permite tabelas flutuarem em qualquer posição
\usepackage[alf,abnt-repeated-title-omit=yes,abnt-emphasize=bf,abnt-etal-list=0]{abntex2cite} %define os parâmetros necessários para as referências e citações de acordo com a ABNT
\usepackage{amsthm}
\usepackage{amsmath} % Pacote que permite a elaboração de Equações matemáticas
\usepackage[table,xcdraw]{xcolor}
\usepackage{booktabs}
\usepackage{amsfonts}
\usepackage[version=3]{mhchem} % Pacote que permite a inserção de equações e fórmulas químicas em texto
\usepackage{subcaption} % Pacote que permite que imagens possam ser inseridas em composições de imagens lado a lado com o \subfigure
\usepackage{amssymb} %4 Pacotes responsáveis pela formatação matemática
\usepackage{rotating} %permite que textos, figuras e tabelas possam ser rotacionados
\usepackage{changepage}
\usepackage{pgffor} % lista de taelas e figuras
\usepackage[table,xcdraw]{xcolor}
\documentclass[xcolor=table]{beamer}
\renewcommand*{\insertchapterspace}{%
	\addtocontents{lof}{\protect\addvspace{10pt}}%
	\addtocontents{lot}{\protect\addvspace{10pt}} }
\usepackage{fancyhdr}


\begin{document}
%! Author = alanmiranda
%! Date = 14/11/2022
\thispagestyle{empty}
    \begin{center}
        \parbox{3cm}{\includegraphics[scale=1]{pictures/logo_ufpa}}\\
        \vspace{1cm}
        \Large \uppercase{Universidade Federal do Pará}\\
        \Large \uppercase{Instituto de Tecnologia - ITEC}\\
        \vspace{3cm}
        \Large \uppercase{Faculdade de Engenharia Mecânica - FEM}\\
        \Large \uppercase{Laboratório de Química Analítica Quantitativa - LQAQ}\\
        \vspace{3cm}
        \Large \textbf{\uppercase {Relatório de Prática 7: Utilização de Indicadores Ácidos e Bases}} \\
        \Large \textbf{\uppercase {PROF. DR. Carlos Antônio Neves}} \\
        \vspace{3cm}
        \Large \uppercase {Alan Henrique Pereira Miranda - 202102140072}\\
        \Large \uppercase {Gabriel Cruz de Oliveira - 202102140055}\\
        \Large \uppercase {Paloma Gama da Silva - 202102140029}\\
        \Large \uppercase {Silvio Farias Leal - 202102140035}\\
        \vspace{1cm}
        \Large {Belém-PA \\ 2022}

    \end{center}


\pagestyle{empty}
\input{tex/resumo}
\thispagestyle{empty}
\begin{center}
    \listoffigures 				% lista de figuras (opcional)
    \newpage
    \listoftables 				% lista de tabelas (opcional)
    \newpage
    \tableofcontents 			% sumário
    \newpage						%
\end{center}

\cleardoublepage			%
\pagestyle{fancy}			% formatação para corpo do texto
						%
    \indent A manipulação de ácidos e bases são parte do cotidiano de um químico, e é importante que o mesmo saiba como identificar a presença de um ácido ou base em uma solução, para que possa tomar as medidas de segurança necessárias.\  A utilização de indicadores ácidos e bases é uma forma de identificar a presença de um ácido ou base em uma solução, sem a necessidade de realizar uma titulação.\  Os indicadores ácidos e bases são substâncias que mudam de cor quando expostas a uma solução ácida ou básica, respectivamente.\  A cor de um indicador ácido ou base é chamada de cor de transição, e é a cor que o indicador apresenta quando exposto a uma solução ácida ou básica.\  A cor de transição de um indicador ácido ou base é determinada por sua estrutura química, e é uma característica que não pode ser alterada pelo pH da solução.
    \indent O objetivo deste relatório é identificar a presença de ácidos e bases em soluções, utilizando indicadores ácidos e bases, e determinar a cor de transição de cada indicador utilizado, seguindo as orientações do material de apoio do professor.




		%
\section{Fundamentação Teórica}
O estudo do comportamento de molas é uma área importante para diversas áreas relacionadas a física e a engenharia, pois,
diversos materiais demonstram propriedade elástica em seu comportamento natural, como por exemplo, aço, borracha, madeira, etc.

Todo material que sofre deformação quando submetido a uma força externa e retorna a sua forma original quando a força é removida é chamado de material elástico.
A deformação elástica é reversível, ou seja, quando a força é removida, o material retorna a sua forma original. para além disso, consideramos que a deformação é
plástica ou elastoplástica. Quando um determinado material atinge tal condição, ele não retorna a sua forma original, mesmo após a remoção da força externa.

Tal característica é muito estudada dentro da engenharia, pois, a partir do conhecimento do comportamento de molas, é possível criar mecanismos que utilizem
tal propriedade em benefício do projeto, como por exemplo, amortecedores, sistemas de suspensão, sistemas de frenagem, etc.

A aproximação teórica do comportamento das molas é feita através de análises matemáticas, sendo a mais conhecida, a Lei de Hooke, que relaciona a força aplicada
em uma mola com a deformação sofrida por ela. Sendo ela uma aproximação linear do comportamento de molas, ou seja, ela só é válida para pequenas deformações.

\subsection{Lei de Hooke}
A Lei de Hooke é uma aproximação linear do comportamento de molas, ou seja, ela só é válida para pequenas deformações. A Lei de Hooke é dada pela seguinte equação:

\begin{equation}
    F = k \cdot \Delta L
\end{equation}

Onde:
\begin{enumerate}
    \item k : constante elástica da mola (N/m)
    \item $\Delta$L : deformação da mola (m)
    \item F : força aplicada na mola (N)
\end{enumerate}

Quando tratamos de um corpo rígido, podemos calcular a constante elástica, como o quociente entre a força aplicada e a deformação da mola, ou seja:

\begin{equation}
    k = \frac{F}{\Delta L}
\end{equation}

Mas esta abordagem considera apenas fatores geométricos, como o comprimento da mola e a deformação sofrida por ela, não dizendo muito sobre a disposição geométrica e como isso afeta o comportamento da mola, como no caso de molas helicoidais.

\subsection{Molas Helicoidais}
As molas helicoidais são as mais utilizadas em sistemas mecânicos, pois, são fáceis de fabricar e possuem uma grande variedade de aplicações.
A mola helicoidal é um dispositivo mecânico que possui a capacidade de armazenar energia mecânica, quando submetida a uma força externa, e liberá-la quando a força é removida, porém, sua geometria, composição e características construtivas buscam otimizar tal comportamento elástico, adequando-se a sua aplicação comercial/industrial.

\begin{figure}[h]
	\centering
	\includegraphics[width=0.7\linewidth]{"pics/geometria da mola"}
	\caption{Esquemático: Componentes Geométricos de uma Mola}
	\label{fig:geometria-da-mola}
\end{figure}



A mola helicoidal é composta por um fio de seção transversal circular, enrolado em forma de hélice, com um diâmetro de espira, diâmetro do fio, e com as extremidades retificadas, para que possam ser fixadas em um sistema mecânico.
Tais propriedades são levadas em consideração para estimar com precisão, o valor da constante elástica da mola.


A constante elástica de uma mola é dada pela seguinte equação:

\begin{equation}
    k = \frac{G \cdot d^4}{8 \cdot D^3 \cdot n}
\end{equation}

Onde:

\begin{enumerate}
    \item G : módulo de elasticidade transversal do material (Pa)
    \item d : diâmetro do fio (m)
    \item D : diâmetro médio da mola (m)
    \item n : número de espiras
    \item k : constante elástica da mola (N/m)
\end{enumerate}



    	% inserir os capítulos do seu
\newpage


\section{Metodologia}
Para a realização do experimento, foi utilizado um dinamômetro, um paquímetro, uma mola helicoidal, um suporte para a mola, um suporte para o dinamômetro e um suporte para o paquímetro.
O experimento foi realizado em 4 etapas, sendo elas:

\begin{enumerate}
    \item Medição do diâmetro do fio.
    \item Medição do diâmetro médio da mola.
    \item Medição do número de espiras.
    \item Medição da relação entre a força aplicada e a deformação da mola.
\end{enumerate}

\subsection{Medições de Parâmetros Geométricos}
O diâmetro do fio foi medido utilizando um paquímetro, com uma resolução de 0,05 mm, e foi medido em 3 pontos diferentes, para garantir a precisão da medição.
Também foram medidas as dimensões da mola, como o diâmetro externo, o diâmetro interno e a altura da mola, para que fosse possível calcular o diâmetro médio da mola.
Por fim, foi contado o número de espiras da mola, chegando na seguinte relação:

% Please add the following required packages to your document preamble:
% \usepackage{graphicx}
\begin{center}
    \begin{table}[h]
        \centering
        \begin{tabular}{|c|c|c|c|}
            \rowcolor[HTML]{DAE8FC}
            \hline
            index & d\_ext (mm) & d\_int (mm) & d\_fio (mm) \\ \hline
            t1   & 44.4        & 37.8        & 2.51        \\ \hline
            t2   & 44.33       & 37.87       & 2.53        \\ \hline
            t3   & 44.35       & 37.82       & 2.49        \\ \hline
            t4   & 44.42       & 37.79       & 2.51        \\ \hline
            t5   & 44.4        & 37.77       & 2.5         \\ \hline
        \end{tabular}
        \caption{Dados obtidos no experimento.}
        \label{tab:dados_adquiridos}
        \textbf{Fonte:} Elaborado pelo autor (2023).
    \end{table}
\end{center}
		% trabalho.
\section{Conclusão}\label{sec:conclusao}
        \indent O experimento foi realizado com sucesso, e os resultados obtidos foram dentro de uma margem de esperada, considerando a disponibilidade ferramental e técnica.\ A reação de neutralização das soluções aquosas de hidróxido de sódio e ácido clorídrico foi realizada com sucesso, e o calor de reação foi de $54,7 kJ/mol$.\ A reação de neutralização ácido-base mas com o hidróxido de sódio no estado sólido, foi realizada com sucesso, e o calor de reação foi de $98,30 kJ/mol$.\ Tais reações tornaram possível estimar o calor de solubilização do hidróxido de sódio com sucesso, e ao utilizar a Lei de Hess, calor de solubilização foi de $43,7 kJ/mol$.\\

        \indent A realização dos experimentos foi de grande proveito para equipe, pois permitiu a imersão dentro das atividades práticas de laboratório, e a aplicação de conceitos teóricos em situações reais, o que permitiu a aquisição de novos conhecimentos e a revisão de conceitos já estudados.\\		%


\end{document}