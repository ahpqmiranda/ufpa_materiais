\section{Questão:12-31}

Nesta questão, analisamos o movimento de uma partícula ao longo de uma curva circular com raio \(r = 300 \, \text{m}\). A aceleração tangencial da partícula é variável e descrita pela equação \(a_t = -0.001 \cdot s\), onde \(s\) é a posição ao longo do arco em metros. Sabemos que a velocidade da partícula no ponto \(A\) (\(s = 0\)) é \(v_A = 25 \, \text{m/s}\). Nosso objetivo é determinar a velocidade da partícula no ponto \(B\) (\(s = r = 300 \, \text{m}\)).

\subsection*{Equação do Movimento}
A equação do movimento é dada por:
\[
a_t = v \cdot \frac{dv}{ds},
\]
onde:
\begin{itemize}
    \item \(a_t = -0.001 \cdot s\): Aceleração tangencial variável;
    \item \(v\): Velocidade escalar da partícula;
    \item \(s\): Posição ao longo do arco.
\end{itemize}

Substituímos \(a_t\) na equação:
\[
-0.001 \cdot s = v \cdot \frac{dv}{ds}.
\]

Reorganizando:
\[
v \cdot dv = -0.001 \cdot s \cdot ds.
\]

\subsection*{Integração}
Integramos ambos os lados para determinar \(v\) em função de \(s\). No ponto \(A\), temos \(v = v_A = 25 \, \text{m/s}\) quando \(s = 0\):
\[
\int_{v_A}^{v} v \, dv = \int_{0}^{s} -0.001 \cdot s \, ds.
\]

Resolvendo a integral do lado esquerdo:
\[
\frac{v^2}{2} \bigg|_{v_A}^{v} = -0.001 \cdot \frac{s^2}{2} \bigg|_{0}^{s}.
\]

Substituímos os limites:
\[
\frac{v^2}{2} - \frac{v_A^2}{2} = -0.001 \cdot \frac{s^2}{2}.
\]

Reorganizando para \(v^2\):
\[
v^2 = v_A^2 - 0.001 \cdot s^2.
\]

\subsection*{Velocidade no Ponto \(B\)}
No ponto \(B\), \(s = r = 300 \, \text{m}\). Substituímos \(s = 300 \, \text{m}\) e \(v_A = 25 \, \text{m/s}\) na equação:
\[
v^2 = 25^2 - 0.001 \cdot (300)^2.
\]

Calculando:
\[
v^2 = 625 - 0.001 \cdot 90000.
\]

\[
v^2 = 625 - 90 = 535.
\]

A velocidade no ponto \(B\) é:
\[
v = \sqrt{535} \approx 23.13 \, \text{m/s}.
\]

\subsection*{Resultado Final}
A velocidade da partícula no ponto \(B\) (\(s = r = 300 \, \text{m}\)) é:
\[
v \approx 23.13 \, \text{m/s}.
\]
