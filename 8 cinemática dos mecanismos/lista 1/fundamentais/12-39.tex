\section{Questão 12-39}

Nesta questão, analisamos um sistema de polias em que a velocidade de um ponto \(A\) é relacionada à velocidade de um bloco \(D\) devido à configuração do sistema de cordas. Determinamos a velocidade de \(D\) (\(v_D\)) quando a velocidade de \(A\) (\(v_A\)) é \(3 \, \text{m/s}\).

\subsection*{Relação entre as Velocidades}
No sistema de polias, a velocidade de \(A\) é o dobro da velocidade de \(D\), pois a corda se divide em duas partes móveis conectadas ao bloco \(D\). Assim, temos a relação:
\[
v_A = 2 \cdot v_D,
\]
onde:
\begin{itemize}
    \item \(v_A\): Velocidade no ponto \(A\);
    \item \(v_D\): Velocidade no bloco \(D\).
\end{itemize}

\subsection*{Cálculo de \(v_D\)}
Substituímos \(v_A = 3 \, \text{m/s}\) na equação:
\[
3 = 2 \cdot v_D.
\]

Resolvendo para \(v_D\):
\[
v_D = \frac{3}{2} = 1.5 \, \text{m/s}.
\]

\subsection*{Resultado Final}
A velocidade do bloco \(D\) é:
\[
v_D = 1.5 \, \text{m/s}.
\]
