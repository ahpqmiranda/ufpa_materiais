\section{Questão 12-40}

Nesta questão, analisamos um sistema de polias em que a velocidade no ponto \(B\) é relacionada à velocidade do bloco \(A\) devido à configuração do sistema de cordas. Determinamos a velocidade de \(A\) (\(v_A\)) quando a velocidade de \(B\) (\(v_B\)) é \(6 \, \text{m/s}\).

\subsection*{Relação entre as Velocidades}
No sistema de polias, a velocidade no ponto \(B\) é o dobro da velocidade do bloco \(A\), pois a corda se divide em duas partes móveis conectadas ao bloco \(A\). Assim, temos a relação:
\[
v_B = 2 \cdot v_A,
\]
onde:
\begin{itemize}
    \item \(v_B\): Velocidade no ponto \(B\);
    \item \(v_A\): Velocidade no bloco \(A\).
\end{itemize}

\subsection*{Cálculo de \(v_A\)}
Substituímos \(v_B = 6 \, \text{m/s}\) na equação:
\[
6 = 2 \cdot v_A.
\]

Resolvendo para \(v_A\):
\[
v_A = \frac{6}{2} = 3 \, \text{m/s}.
\]

\subsection*{Resultado Final}
A velocidade do bloco \(A\) é:
\[
v_A = 3 \, \text{m/s}.
\]
