\documentclass[a4paper, 11pt]{article}
\usepackage[left=3cm, top=3cm, right=2cm, bottom=2cm]{geometry}
%pacote que define as margens
\usepackage{helvet} % carrega o documento com Fonte ARIAL
\renewcommand{\familydefault}{\sfdefault} % define a fonte ARIAL como padrão do texto
\usepackage[utf8]{inputenc} %pacote que normaliza a acentuação
\usepackage[T1]{fontenc}
\usepackage[brazil]{babel} %pacote que define o idioma
\usepackage{graphicx} %pacote que permite a inserção de imagens
\pagestyle{myheadings} %pacote que define o cabeçalho com o número de página no canto superior direito
\usepackage{setspace} %pacote que permite especificar o espaçamento entre linhas no documento
\usepackage{indentfirst} %pacote que faz indentação na primeira linha do parágrafo
\setlength{\parindent}{1cm} %pacote que define o tamanho da indentação
\usepackage{float} %Pacote que permite tabelas flutuarem em qualquer posição
% \usepackage[alf,abnt-repeated-title-omit=yes,abnt-emphasize=bf,abnt-etal-list=0]{abntex2cite} %define os parâmetros necessários para as referências e citações de acordo com a ABNT
\usepackage{amsthm}
\usepackage{amsmath}
\usepackage{amsfonts}
\usepackage{subcaption} % Pacote que permite que imagens possam ser inseridas em composições de imagens lado a lado com o \subfigure
\usepackage{amssymb} %4 Pacotes responsáveis pela formatação matemática
\usepackage{rotating} %permite que textos, figuras e tabelas possam ser rotacionados
\usepackage{array}    % permite e vetorização de tabelas
\usepackage{longtable} % permite a elaboração de tabelas complexas
\usepackage{colortbl} % permite a utilização de paletas de cores
\usepackage{tikz} % criação de objetos vetoriais
\usetikzlibrary{shapes, arrows.meta, positioning}
\usepackage{pdflscape} % permite alterar o estilo da pagina entre portrato e paisagem
\usepackage{hyperref}
\hypersetup{
colorlinks=true,
linkcolor=darkblue,
filecolor=magenta,
urlcolor=blue,
citecolor=black,
pdftitle={mybib},
pdfpagemode=FullScreen
}
\usepackage{sectsty}
\sectionfont{\fontsize{12}{12}\selectfont}
\subsectionfont{\fontsize{12}{12}\selectfont}
\subsubsectionfont{\fontsize{12}{12}\selectfont}
\usepackage{url}
\usepackage[table,xcdraw]{xcolor}
\usepackage{cite}
\usepackage[brazilian,hyperpageref]{backref}	% Paginas com as citações na bibl
\usepackage[num,overcite]{abntex2cite} % Citações padrão ABNT


\usepackage{pgfgantt}  % gera gantt em latex
\usepackage{acronym} 
%\usepackage[acronym]{glossaries} % cria glossário de siglas
%\makeglossaries

\usepackage[most]{tcolorbox}
\usepackage{listings}        % Para listar códigos
\lstset{
	inputencoding=utf8,
	extendedchars=true,
	literate={á}{{\'a}}1 {é}{{\'e}}1 {í}{{\'i}}1 {ó}{{\'o}}1 {ú}{{\'u}}1
	{Á}{{\'A}}1 {É}{{\'E}}1 {Í}{{\'I}}1 {Ó}{{\'O}}1 {Ú}{{\'U}}1
	{à}{{\`a}}1 {è}{{\`e}}1 {ì}{{\`i}}1 {ò}{{\`o}}1 {ù}{{\`u}}1
	{À}{{\`A}}1 {È}{{\`E}}1 {Ì}{{\`I}}1 {Ò}{{\`O}}1 {Ù}{{\`U}}1
	{ã}{{\~a}}1 {õ}{{\~o}}1 {Ã}{{\~A}}1 {Õ}{{\~O}}1
	{â}{{\^a}}1 {ê}{{\^e}}1 {î}{{\^i}}1 {ô}{{\^o}}1 {û}{{\^u}}1
	{Â}{{\^A}}1 {Ê}{{\^E}}1 {Î}{{\^I}}1 {Ô}{{\^O}}1 {Û}{{\^U}}1
	{ç}{{\c{c}}}1 {Ç}{{\c{C}}}1
}

\usepackage{xcolor}

\definecolor{codegreen}{rgb}{0,0.6,0}
\definecolor{codegray}{rgb}{0.5,0.5,0.5}
\definecolor{codepurple}{rgb}{0.58,0,0.82}
\definecolor{backcolour}{rgb}{0.95,0.95,0.92}

\lstdefinestyle{mystyle}{
	backgroundcolor=\color{backcolour},   
	commentstyle=\color{codegreen},
	keywordstyle=\color{magenta},
	numberstyle=\tiny\color{codegray},
	stringstyle=\color{codepurple},
	basicstyle=\ttfamily\footnotesize,
	breakatwhitespace=false,         
	breaklines=true,                 
	captionpos=b,                    
	keepspaces=true,                 
	numbers=left,                    
	numbersep=5pt,                  
	showspaces=false,                
	showstringspaces=false,
	showtabs=false,                  
	tabsize=2
}

\lstset{style=mystyle}


\begin{document}  
    % Deve conter o título do trabalho, nome do autor, nome da instituição, curso, orientador, local e ano. A capa deve seguir as normas da instituição para formatação (ABNT ou outra específica).
	\thispagestyle{empty}

\begin{center}
	\begin{figure}[h]
  \centering
		\includegraphics[width=0.21\linewidth]{imagens/ufpa.png}
		\label{fig:ufpa}
	\end{figure}
	
	
	\vspace{1cm}
	\large \uppercase{UNIVERSIDADE FEDERAL DO PARÁ}\\
	\large \uppercase{INSTITUTO DE TECNOLOGIA}\\
	\large \uppercase{BACHARELADO EM ENGENHARIA MECÂNICA}\\
	\vspace{6cm}
	\large \uppercase{TRANSFERÊNCIA DE CALOR E MASSA II}\\
	\vspace{1cm}
	\large \uppercase {AVALIAÇÃO FINAL: PROJETO DE DIMENSIONAMENTO DE TROCADORES DE CALOR} \\
	\vspace{7cm}
	\large {BELÉM/PA \\ 2024}

 \newpage
 \thispagestyle{empty}
 \large \uppercase{alan henrique pereira miranda - 202102140072}\\
 \large \uppercase{GABRIEL ASSUNÇÃO DOS SANTOS - 201902140024}\\
 \large \uppercase{LEILANE MARIA RIBEIRO NOGUEIRA - 201902140023}\\
 \large \uppercase{VINÍCIUS LIMA SILVA - 201902140032	}\\
 \large \uppercase{WESLEY DAVIDSON MESQUITA GARROS - 201902140020}\\
 \vspace{3cm}
	\large \uppercase{TRANSFERÊNCIA DE CALOR E MASSA II}\\
\vspace{1cm}
\large \uppercase {AVALIAÇÃO FINAL: PROJETO DE DIMENSIONAMENTO DE TROCADORES DE CALOR} \\
 \vspace{1cm}
 \singlespacing
 \hspace{8cm} % posicionando a caixa de texto
 \begin{minipage}{7cm}
	Projeto referente ao tópico Trocadores de Calor, como requisito avaliativo da disciplina Transferência de Calor e Massa II, lecionada na Universidade Federal do Pará. \\
	
	Profa. Dra.: Danielle Regina da Silva Guerra
	\vspace{1cm}
	
	Belém-PA 26 de outubro de 2024
	\vspace{4cm}
\end{minipage}

\onehalfspacing
\begin{center}
	
	EXAMINADOR\\
	\vspace{3cm}
	\rule{10cm}{0.15mm} \\
	Profa. Dra.: Danielle Regina da Silva Guerra\\
	Universidade Federal do Pará - UFPA
\end{center}
\newpage

\begin{center}
    % BLOCO DE FIGURAS
    \thispagestyle{empty}
	\listoffigures
	\newpage
    % SUMARIO
    \thispagestyle{empty}
    \tableofcontents

\end{center}

\newpage
\thispagestyle{empty}
	
\end{center}
    % Resumo, dedicatórias etc (elementos pré extuais)


    % elementos textuais
    % -> introdução: Conteúdo: Apresenta o tema do trabalho, o problema a ser investigado, os objetivos gerais e específicos, a justificativa para a realização do estudo e uma breve descrição da estrutura do documento. Deve contextualizar o leitor sobre a relevância do trabalho.
	\section*{Introdução}

Este solucionário tem como objetivo apresentar a resolução detalhada das questões propostas na lista de exercícios do capítulo 12 do livro "Dinâmica", 12\textordfeminine edição. As questões abrangem os tópicos fundamentais e avançados relacionados à cinemática de mecanismos, com enfoque em problemas práticos e teóricos.\\

A lista é composta por um total de 44 questões, divididas em dois grupos: problemas fundamentais e problemas gerais, exigindo o domínio dos conceitos apresentados no capítulo. As resoluções seguem os procedimentos sugeridos pelo autor do livro, promovendo clareza e rigor matemático para facilitar a compreensão dos conceitos e métodos aplicados.\\

Espera-se que este material contribua para o aprendizado e a consolidação dos conteúdos estudados, além de servir como um guia para a resolução de problemas similares.\\


    % -> justificativas:  Explique a importância do tema escolhido, destacando o motivo pelo qual o trabalho foi desenvolvido. Aborde o impacto acadêmico, científico, tecnológico ou social do estudo. A justificativa deve convencer o leitor da relevância do projeto.
    \input{2. textuais/1.1 - justificativas}

    % objetivos: Liste e descreva os objetivos gerais e específicos do trabalho. O objetivo geral deve refletir o propósito principal do estudo, enquanto os específicos detalham as etapas que serão necessárias para atingir o objetivo geral.
    \subsection{Objetivos}

\begin{itemize}
	
	\item Explorar os princípios fundamentais da transferência de calor e os diferentes mecanismos envolvidos nos trocadores de calor;
	\item Elaborar um memorial de cálculo em planilha eletrônica com base nos fundamentos de seleção e dimensionamento de trocadores de calor para a seleção de um trocador de calor a partir de uma condição pré-determinada de projeto;
	\item Utilizar o método da eficiência – NTU para dimensionamento e seleção de um tipo de trocador de calor, de acordo com o catálogo do fabricante oferecido;	 
	\item Explorar os princípios fundamentais da transferência de calor e os diferentes mecanismos envolvidos nos trocadores de calor;
	\item Elaborar um memorial de cálculo em planilha eletrônica com base nos fundamentos de seleção e dimensionamento de trocadores de calor para a seleção de um trocador de calor a partir de uma condição pré-determinada de projeto;
	\item Utilizar o método da eficiência – NTU para dimensionamento e seleção de um tipo de trocador de calor, de acordo com o catálogo do fabricante oferecido;
\end{itemize}

    % requisitos: Defina os requisitos funcionais e não funcionais que o projeto deve atender. Descreva as necessidades e especificações técnicas que o sistema, software ou pesquisa deve cumprir.
    \section{Metodologia}
Para os trocadores de calor, diversos métodos têm sido publicados por diferentes autores ao longo dos anos. Os mais tradicionais e amplamente utilizados são os da média logarítmica da diferença de temperaturas (MLDT) e efetividade – NUT (εNUT). Basicamente, estes métodos exploram a razão entre a taxa de transferência de calor real e a taxa máxima possível. Posteriormente, outras linhas de pesquisa foram abordadas, como a análise de geração de entropia por SeКulic (1990), o conceito de eficiência por \cite{orgeda_2020_trocadores} e o método do princípio da uniformidade do campo da diferença de temperaturas proposto \cite{cengel_2012_transferencia}

O método da efetividade – NUT (ε-NUT) é amplamente utilizado em situações onde o tamanho do trocador de calor e as temperaturas de entrada são conhecidos e a taxa de transferência de calor e as temperaturas de saída dos fluidos são pretendidas. Problemas de dimensionamento também podem ser solucionados através deste método. Em resumo, o método ε-NUT pode ser definido como a razão entre a taxa de transferência de calor real do trocador de calor em estudo e a taxa de transferência máxima, que pode ser obtida de um trocador de calor contracorrente puro com comprimento infinito, o que garante a máxima diferença possível de temperaturas no fluido de menor capacidade térmica. Esta razão pode então ser escrita da seguinte forma: \cite{stenstrasser_2017_projeto, abdallah_2018_multi}

\begin{equation}
	\varepsilon = \frac{q}{q_{\text{máx}}} = \frac{C_q (T_{q,\text{ent}} - T_{q,\text{sai}})}{C_{\text{min}}(T_{q,\text{ent}} - T_{f,\text{ent}})} = \frac{C_f (T_{f,\text{sai}} - T_{f,\text{ent}})}{C_{\text{min}}(T_{q,\text{ent}} - T_{f,\text{ent}})}
\end{equation}
De acordo com \citen{abdallah_2018_multi}, a efetividade de todo trocador de calor pode ser expressa em termos do número de unidades de transferência (NUT) e da razão entre as capacidades térmicas dos fluidos C*. 

A partir de então, fórmulas específicas foram desenvolvidas para os principais tipos e arranjos de trocadores de calor, como pode ser consultado em \citenum{stenstrasser_2017_projeto} para casos com escoamentos em paralelo, contracorrente, tipo casco-tubo e com escoamentos cruzados, objeto de estudo do presente em \citen{stenstrasser_2017_projeto} encontram-se dois métodos para a análise dos trocadores de calor: média logarítmica das diferenças de temperatura e o método da efetividade. 

Para calcular o desempenho de um trocador é necessário relacionar a taxa total de transferência de calor, as temperaturas de entrada e saída dos fluidos, o coeficiente global de transferência de calor e a área total disponível para a troca térmica. A taxa de transferência de calor pode ser determinada pelas equações abaixo:
\\
\begin{equation}
	q = m_q \cdot C_{p,q} (T_{q,e} - T_{q,s})
\end{equation}

\begin{equation}
	q = m_f \cdot C_{p,f} (T_{f,e} - T_{f,s})
\end{equation}

Onde os índices \(q \) e \(f\), referem-se aos fluidos quentes e frios, \(e\) e \(s\) representam a entrada e a saída respectivamente. As vazões das correntes são definidas como \(m\), \(C_p\) é a capacidade calorífica e \(T\) as temperaturas.
A diferença de temperatura (\(\Delta T\)) entre os fluidos quente e frio é dada por:

\begin{equation}
	\Delta T \equiv T_q - T_f
\end{equation}

\begin{equation}
	\Delta T_1 = T_{q,e} - T_{f,s}
\end{equation}

\begin{equation}
	\Delta T_2 = T_{q,s} - T_{f,e}
\end{equation}

Quando as temperaturas não são conhecidas ou especificadas, é preferível utilizar o método da efetividade-NUT ($\varepsilon -$  NUT). Nesse método, calcula-se primeiramente o máximo calor trocado. Para isso tem-se:


\begin{equation}
	C_f < C_q \quad q_{\text{máx}} = C_f (T_{q,i} - T_{f,i})
\end{equation}

\begin{equation}
	C_q < C_f \quad q_{\text{máx}} = C_q (T_{q,i} - T_{f,i})
\end{equation}


Onde \(C_f\) e \(C_q\) e são as vazões mássicas multiplicadas pelo calor específico correspondente dos fluidos frio e quente, respectivamente.
Com base nas equações anteriores, podemos escrever a seguinte equação:

\begin{equation}
	q_{max} = C_{min}  (T_{h,i} - T_{c,i})
\end{equation}

O conhecimento da efetividade é útil, pois a taxa real de transferência de calor pode ser determinada de imediato:

\begin{equation}
	q = \varepsilon C_{min} (T_{q,e} - T_{f,e})
\end{equation}

O número de unidades de transferência (NUT) é um parâmetro adimensional utilizado para a análise de trocadores de calor, definido como:

\begin{equation}
	\text{NUT} = \frac{UA}{C_{min}}
\end{equation}

Foram desenvolvidas equações que determinam de forma específica a relação efetividade-NUT. Para trocadores de calor duplo tubo em paralelo tem-se:\\

\textbf{A) Parallel flow}
\[
\varepsilon = \frac{1 - \exp \left[ -\text{NTU} (1 + C_r) \right]}{1 + C_r}
\]

\textbf{B) Counterflow}
\[
\varepsilon = \frac{1 - \exp \left[ -\text{NTU} (1 - C_r) \right]}{1 - C_r \exp \left[ -\text{NTU} (1 - C_r) \right]} \quad (C_r < 1)
\]

Onde, 

\begin{equation}
	C_r = \frac{C_{min}}{C_{max}}
\end{equation}

No caso de contracorrente:

\[
\varepsilon = \frac{1 - \exp \left[-\text{NUT} (1 - C_r)\right]}{1 - C_r \exp \left[-\text{NUT} (1 - C_r)\right]} \quad (C_r < 1)
\]

\[
\varepsilon \equiv \frac{\text{NUT}}{1 + \text{NUT}} \quad (C_r = 1)
\]


Em cálculos envolvendo o projeto de trocadores de calor contracorrente, é mais conveniente trabalhar com relação ε-NUT na forma:
\begin{equation}
	\text{NUT} = \frac{1}{C_r - 1} \ln \left( \frac{\varepsilon - 1}{\varepsilon \cdot C_r - 1} \right) \quad (C_r < 1)
\end{equation}

\begin{equation}
	\text{NUT} \equiv \frac{\varepsilon}{1 - \varepsilon} \quad (C_r = 1)
\end{equation}

    
    	% metodologia: Descreva a metodologia utilizada para a realização do trabalho. Detalhe os procedimentos, técnicas e ferramentas empregadas na pesquisa, desenvolvimento ou análise do projeto.
    \section{Desenvolvimento Prático}

Para garantir um processo documentado e reprodutível, o memorial de cálculo foi estruturado utilizando Python, uma linguagem de programação amplamente adotada para análises científicas e de engenharia devido à sua flexibilidade, bibliotecas especializadas e capacidade de automação de cálculos complexos. Python permitiu que os autores mapeassem detalhadamente o desenvolvimento teórico do trocador de calor, criando rotinas de cálculo que facilitam a verificação dos resultados e asseguram que todas as hipóteses e passos intermediários estejam explicitamente documentados. A escolha por Python também contribuiu para a criação de um fluxo de trabalho modular, onde cada cálculo ou etapa teórica do desenvolvimento do trocador de calor foi representado por funções e scripts específicos.

Dentro do memorial, cada seção aborda um aspecto do desenvolvimento do trocador de calor, desde os cálculos preliminares para dimensionamento até a verificação de parâmetros operacionais, como eficiência térmica e transferência de calor. A implementação dos cálculos em Python proporcionou uma maneira prática de estruturar o memorial, permitindo o ajuste de parâmetros em tempo real e facilitando a análise de sensibilidade e a validação de resultados para diferentes condições de operação.

Para validar a integridade dos cálculos e garantir a reprodutibilidade dos procedimentos, os autores aplicaram uma série de testes e verificações automatizadas que permitiram revisar o memorial em busca de inconsistências ou erros. Esta abordagem sistemática, centrada na programação, fornece um documento detalhado, onde cada etapa é justificável e os resultados podem ser replicados, contribuindo significativamente para a transparência e confiabilidade do projeto.

\subsection{Código Comentado}
No desenvolvimento do experimento, a decisão de se usar python se baseou na familiaridade com seus recursos e ao fato de ser uma ferramenta amplamente reconhecida na área de ciência de dados. Foram empregadas diversas bibliotecas especializadas para realizar os cálculos necessários e assegurar a precisão dos resultados. 

Começamos pela importação dos frameworks para processamento:

\begin{center}
	\begin{lstlisting}[language=Python]
		import numpy as np  # para cálculos matemáticos
		import pandas as pd # para manipulação de dados
		import matplotlib.pyplot as plt # para plotar gráficos
\end{lstlisting}

\text{Fonte: Elaborado pelos autores (2024)}
\end{center}

A biblioteca NumPy foi utilizada para operações vetoriais, enquanto o Pandas facilitou o processamento e manipulação dos dados, principalmente na forma de tabelas (DataFrames). sendo essencial para a leitura, filtragem, limpeza, e agregação de dados. \cite{g_2008_manipulating, harris_2020_array}

Com o Matplotlib, a equipe construiu gráficos que permitiram a visualização clara das informações, tanto em valores do Sistema Internacional (SI) quanto em valores de engenharia.  \cite{hunter_2007_matplotlib}\\

Declaração de variáveis:
\begin{center}
	\begin{lstlisting}[language=python]
		# Dados do problema
		T_agua_entrada = 20 # Temperatura de entrada da água, em °C
		T_agua_saida = 26		# Temperatura de saída da água, em °C
		T_diesel_in = 65		# Temperatura da entrda do diesel, em °C
		T_diesel_out = 38		# Temperatura da saida do diesel, em ºC
		m_dot_agua = 1.5		# Vazão mássica da água, em kg/s
		m_dot_diesel = .7		# Vazão mássica do diesel, em kg/s
		rho_diesel = 0.850		# kg/l
		v_dot_diesel = m_dot_diesel / rho_diesel * 60 # vazão volumétrica do disel
		D_interno = 0.014		# Diâmetro interno do tubo, em metros
		U = 640		# Coeficiente global de transferência de calor, em W/m²·K
		# Calores específicos
		c_p_agua = 4.18		# Calor específico da água, em kJ/kg·K
		c_p_diesel = 2		# Calor específico do óleo diesel, em kJ/kg·K
		# Convertendo calor específico de kJ para J
		c_p_agua *= 1000		# em J/kg·K
		c_p_diesel *= 1000		# em J/kg·K
	\end{lstlisting}
	\text{Fonte: Elaborado pelos autores (2024)}
\end{center}

O trecho a seguir verifica as capacidades térmicas da água e do óleo diesel, para estabelecer a constante \(c\) de dimensionamento de trocas térmicas:

\begin{center}
	\begin{lstlisting}[language=python]
		# Passo 1: Determinar C_h e C_c
		C_h = m_dot_diesel * c_p_diesel # Taxa de capacidade térmica do fluido quente,␣
		↪em W/K
		C_c = m_dot_agua * c_p_agua # Taxa de capacidade térmica do fluido frio, em W/K
	\end{lstlisting}
	\text{Fonte: Elaborado pelos autores (2024)}
\end{center}

A seguir, é realizado o cálculo da constante C, através da verificação das funções máximo e mínimo no python:
\begin{center}
	\begin{lstlisting}[language=python]
		# Identificar o menor valor entre C_h e C_c
		C_min = min(C_h, C_c)
		C_max = max(C_h, C_c)
		c = C_min / C_max
	\end{lstlisting}
	\text{Fonte: Elaborado pelos autores (2024)}
\end{center}

Em seguida, verificamos a taxa máxima de transferência térmica e a taxa real:

\begin{center}
	\begin{lstlisting}[language=python]
		# Passo 3: Calcular a taxa máxima de transferência de calor (Q_max)
		# Temperatura de entrada do fluido quente, em °C
		# Temperatura de entrada do fluido frio, em °C
		Q_max = C_min * (T_diesel_in - T_agua_entrada)
		
		# Passo 4: Calcular a taxa real de transferência de calor (Q)
		Q_real = m_dot_agua * c_p_agua * (T_agua_saida - T_agua_entrada)
	\end{lstlisting}
	\text{Fonte: Elaborado pelos autores (2024)}
\end{center}

A seguir, temos o cálculo da efetividade e do NTU


\begin{center}
	\begin{lstlisting}[language=python]
		# Passo 5: Calcular a efetividade
		efetividade = Q_real / Q_max
		
		# Passo 6: Calcular o NTU usando a fórmula para trocador de calor contracorrente
		NTU = (1 / (c - 1)) * math.log((efetividade - 1) / (efetividade * c - 1))
	\end{lstlisting}
	\text{Fonte: Elaborado pelos autores (2024)}
\end{center}

E por último, calculamos as dimensões de área e comprimento total de tubulação:



\begin{center}
	\begin{lstlisting}[language=python]
		# Passo 7: Calcular a área de transferência de calor (A_s)
		A_s = (NTU * C_min) / U
		# Passo 8: Calcular o comprimento do tubo (L)
		L = A_s / (math.pi * D_interno)
	\end{lstlisting}
	\text{Fonte: Elaborado pelos autores (2024)}
\end{center}

Os resultados obtidos nos cálculos, foram:
\begin{itemize}
	\item $C_{min}: 1400.00 \, \mathrm{W/K}$
	\item $C_{max}: 6270.00 \, \mathrm{W/K}$
	\item $Q_{max}: 63000.00 \, \mathrm{W}$
	\item $Q_{real}: 37620.00 \, \mathrm{W}$
	\item Efetividade: $0.597$
	\item NTU: $0.986$
	\item Área de transferência de calor $(A_s): 2.16 \, \mathrm{m^2}$
	\item Comprimento do tubo $(L): 49.05 \, \mathrm{m}$
\end{itemize}



\begin{center}
	\begin{lstlisting}[language=python]
		conteúdo...
	\end{lstlisting}
	\text{Fonte: Elaborado pelos autores (2024)}
\end{center}

    
    % desenvolvimento: Apresente o desenvolvimento do trabalho, dividindo-o em seções e subseções conforme a necessidade. Descreva as etapas, resultados e discussões do projeto, incluindo gráficos, tabelas, imagens e outros elementos visuais.
    \section{Verificação e Seleção do Trocador de calor}

Os critérios selecionados para análise do trocador de calor foram definidos visando garantir que o projeto atendesse às exigências de desempenho e eficiência térmica. Com base nos resultados obtidos anteriormente, os parâmetros estabelecidos são:

\begin{enumerate}
	\item \textbf{Área de Transferência de Calor}: A área de troca térmica deve ser maior que 2,16 m². Este critério é essencial para assegurar que o trocador de calor tenha capacidade suficiente para realizar a transferência de calor necessária entre os fluidos, atendendo à demanda do processo.
	
	\item \textbf{Vazão de Óleo}: A vazão mínima do óleo deve ser superior a 35,7 L/min. Este valor foi definido para garantir que o fluido tenha uma taxa de fluxo adequada, maximizando a eficiência de troca térmica e prevenindo possíveis pontos de estagnação no sistema.
	
	\item \textbf{Potência Térmica Mínima}: A potência mínima deve ser inferior a 37,6 kW, o valor de \( Q_{real} \) obtido nos cálculos. Este limite inferior garante que o trocador possa operar adequadamente sem ultrapassar a carga mínima necessária para o sistema.
	
	\item \textbf{Potência Térmica Máxima}: A potência máxima deve ser superior a 63 kW, o valor de \( Q_{max} \) calculado. Esse critério assegura que o trocador de calor possui capacidade suficiente para suportar picos de demanda térmica, evitando perda de eficiência em condições de operação máxima.
\end{enumerate}

Esses critérios foram definidos para assegurar a robustez e a eficiência do trocador de calor, considerando tanto a operação regular quanto as possíveis variações no processo. Eles também servem como parâmetros para futuras revisões de desempenho e manutenção do sistema.

Porém, ao comparar os dados com a tabela de modelos disponíveis, verificou-se o seguinte padrão:

\begin{figure}[h]
	\centering
	\caption{Valores da tabela de trocadores de calor, classificados em verde como "ok" para atendimento dos requisitos e em vermelho para "desacordo" com o projeto}
	\label{fig:screenshot001}
	\includegraphics[width=1\linewidth]{imagens/screenshot001}
	
	\text{Fonte: Elaborado pelos autores (2024)}
\end{figure}

Logo, nenhuma das opções demonstrou-se viável para o projeto. Qualquer tentativa de utilizar algum dos trocadores de calor disponíveis, irá requerer alterações em seus projetos ou em adaptações (não recomendadas) em seus padrões de operação que podem vir a diminuir a vida útil e/ou tornar a manutenção mais onerosa do que o necessário.

    
    % resultados: Apresente os resultados obtidos durante a realização do trabalho. Descreva as análises, interpretações e conclusões a partir dos dados coletados, destacando os principais achados e contribuições do estudo.
    \input{2. textuais/4 - testes}
    
    % conclusão: Faça uma síntese dos resultados obtidos, destacando os principais achados e contribuições do trabalho. Apresente as limitações do estudo e sugestões para pesquisas futuras.
    \section{Considerações Finais}

O desenvolvimento do projeto do trocador de calor e os cálculos realizados foram essenciais para compreender as condições operacionais e a demanda térmica exigida pelo sistema. Através desses cálculos, foi possível estabelecer os parâmetros fundamentais para o projeto, como a área de troca térmica, a vazão de óleo e os limites de potência térmica, assegurando que as especificações atendam aos requisitos de eficiência e desempenho esperados.

Entretanto, ao avaliar as opções de trocadores de calor disponíveis, verificou-se que nenhum dos modelos pesquisados atendia integralmente às necessidades do projeto. Este cenário destaca a importância de realizar uma nova pesquisa de mercado para identificar alternativas que possuam especificações mais alinhadas às demandas do sistema. A seleção de um trocador de calor que atenda plenamente aos critérios estabelecidos é essencial para garantir a operação eficiente e sustentável do processo.

Dessa forma, os resultados obtidos até o momento servem como uma base sólida para orientar essa pesquisa de mercado, possibilitando uma comparação mais precisa entre as opções disponíveis e aumentando as chances de encontrar um equipamento que satisfaça todas as exigências do projeto.

\section{Conclusão}

O projeto do trocador de calor, fundamentado em cálculos rigorosos e parâmetros específicos, proporcionou uma compreensão detalhada das necessidades térmicas e operacionais do sistema. Embora os critérios estabelecidos tenham sido cuidadosamente definidos para garantir a eficiência e a confiabilidade do equipamento, a análise revelou que nenhum dos trocadores de calor disponíveis atendia plenamente às demandas do projeto. Portanto, é necessário realizar uma nova pesquisa de mercado para identificar alternativas que melhor correspondam aos requisitos estabelecidos, assegurando que o trocador selecionado seja capaz de operar de forma eficaz nas condições especificadas.

    


	\addcontentsline{toc}{section}{Referências} %adiona a seção das referências no sumário
	\bibliographystyle{abntex2cite}
    \newpage
	\bibliography{referencias}
\end{document}
