\documentclass[a4paper, 11pt]{article}
\usepackage[left=3cm, top=3cm, right=2cm, bottom=2cm]{geometry}
%pacote que define as margens
\usepackage{helvet} % carrega o documento com Fonte ARIAL
\renewcommand{\familydefault}{\sfdefault} % define a fonte ARIAL como padrão do texto
\usepackage[utf8]{inputenc} %pacote que normaliza a acentuação
\usepackage[T1]{fontenc}
\usepackage[brazil]{babel} %pacote que define o idioma
\usepackage{graphicx} %pacote que permite a inserção de imagens
\pagestyle{myheadings} %pacote que define o cabeçalho com o número de página no canto superior direito
\usepackage{setspace} %pacote que permite especificar o espaçamento entre linhas no documento
\usepackage{indentfirst} %pacote que faz indentação na primeira linha do parágrafo
\setlength{\parindent}{1cm} %pacote que define o tamanho da indentação
\usepackage{float} %Pacote que permite tabelas flutuarem em qualquer posição
% \usepackage[alf,abnt-repeated-title-omit=yes,abnt-emphasize=bf,abnt-etal-list=0]{abntex2cite} %define os parâmetros necessários para as referências e citações de acordo com a ABNT
\usepackage{amsthm}
\usepackage{amsfonts}
\usepackage{subcaption} % Pacote que permite que imagens possam ser inseridas em composições de imagens lado a lado com o \subfigure
\usepackage{amssymb} %4 Pacotes responsáveis pela formatação matemática
\usepackage{rotating} %permite que textos, figuras e tabelas possam ser rotacionados
\usepackage{array}    % permite e vetorização de tabelas
\usepackage{longtable} % permite a elaboração de tabelas complexas
\usepackage{colortbl} % permite a utilização de paletas de cores
\usepackage{tikz} % criação de objetos vetoriais
\usepackage{mathtools}

\usetikzlibrary{shapes, arrows.meta, positioning}
\usepackage{pdflscape} % permite alterar o estilo da pagina entre portrato e paisagem
\usepackage{hyperref}
\hypersetup{
colorlinks=true,
linkcolor=black, % alterado para black, antes darkblue por bug
filecolor=magenta,
urlcolor=blue,
citecolor=black,
pdftitle={mybib},
pdfpagemode=FullScreen
}
\usepackage{sectsty}
\sectionfont{\fontsize{12}{12}\selectfont}
\subsectionfont{\fontsize{12}{12}\selectfont}
\subsubsectionfont{\fontsize{12}{12}\selectfont}
\usepackage{url}
\usepackage{pgfplots}
%\usepackage[table,xcdraw]{xcolor}
%\usepackage[brazilian,hyperpageref]{backref}	% Paginas com as citações na bibl
%\usepackage[num,overcite]{abntex2cite} % Citações padrão ABNT
%\usepackage{cite}
%\usepackage{pgfgantt}  % gera gantt em latex
\usepackage{acronym} 
%\usepackage[acronym]{glossaries} % cria glossário de siglas
%\makeglossaries
\usepackage[most]{tcolorbox}
\usepackage{listings}        % Para listar códigos
\lstset{
	inputencoding=utf8,
	extendedchars=true,
	literate={á}{{\'a}}1 {é}{{\'e}}1 {í}{{\'i}}1 {ó}{{\'o}}1 {ú}{{\'u}}1
	{Á}{{\'A}}1 {É}{{\'E}}1 {Í}{{\'I}}1 {Ó}{{\'O}}1 {Ú}{{\'U}}1
	{à}{{\`a}}1 {è}{{\`e}}1 {ì}{{\`i}}1 {ò}{{\`o}}1 {ù}{{\`u}}1
	{À}{{\`A}}1 {È}{{\`E}}1 {Ì}{{\`I}}1 {Ò}{{\`O}}1 {Ù}{{\`U}}1
	{ã}{{\~a}}1 {õ}{{\~o}}1 {Ã}{{\~A}}1 {Õ}{{\~O}}1
	{â}{{\^a}}1 {ê}{{\^e}}1 {î}{{\^i}}1 {ô}{{\^o}}1 {û}{{\^u}}1
	{Â}{{\^A}}1 {Ê}{{\^E}}1 {Î}{{\^I}}1 {Ô}{{\^O}}1 {Û}{{\^U}}1
	{ç}{{\c{c}}}1 {Ç}{{\c{C}}}1
}

%\usepackage{xcolor}

\definecolor{codegreen}{rgb}{0,0.6,0}
\definecolor{codegray}{rgb}{0.5,0.5,0.5}
\definecolor{codepurple}{rgb}{0.58,0,0.82}
\definecolor{backcolour}{rgb}{0.95,0.95,0.92}

\lstdefinestyle{mystyle}{
	backgroundcolor=\color{backcolour},   
	commentstyle=\color{codegreen},
	keywordstyle=\color{magenta},
	numberstyle=\tiny\color{codegray},
	stringstyle=\color{codepurple},
	basicstyle=\ttfamily\footnotesize,
	breakatwhitespace=false,         
	breaklines=true,                 
	captionpos=b,                    
	keepspaces=true,                 
	numbers=left,                    
	numbersep=5pt,                  
	showspaces=false,                
	showstringspaces=false,
	showtabs=false,                  
	tabsize=2
}

\lstset{style=mystyle}


\begin{document}  
    % Deve conter o título do trabalho, nome do autor, nome da instituição, curso, orientador, local e ano. A capa deve seguir as normas da instituição para formatação (ABNT ou outra específica).
	\thispagestyle{empty}

\begin{center}
	\begin{figure}[h]
  \centering
		\includegraphics[width=0.21\linewidth]{imagens/ufpa.png}
		\label{fig:ufpa}
	\end{figure}
	
	
	\vspace{1cm}
	\large \uppercase{UNIVERSIDADE FEDERAL DO PARÁ}\\
	\large \uppercase{INSTITUTO DE TECNOLOGIA}\\
	\large \uppercase{BACHARELADO EM ENGENHARIA MECÂNICA}\\
	\vspace{6cm}
	\large \uppercase{TRANSFERÊNCIA DE CALOR E MASSA II}\\
	\vspace{1cm}
	\large \uppercase {AVALIAÇÃO FINAL: PROJETO DE DIMENSIONAMENTO DE TROCADORES DE CALOR} \\
	\vspace{7cm}
	\large {BELÉM/PA \\ 2024}

 \newpage
 \thispagestyle{empty}
 \large \uppercase{alan henrique pereira miranda - 202102140072}\\
 \large \uppercase{GABRIEL ASSUNÇÃO DOS SANTOS - 201902140024}\\
 \large \uppercase{LEILANE MARIA RIBEIRO NOGUEIRA - 201902140023}\\
 \large \uppercase{VINÍCIUS LIMA SILVA - 201902140032	}\\
 \large \uppercase{WESLEY DAVIDSON MESQUITA GARROS - 201902140020}\\
 \vspace{3cm}
	\large \uppercase{TRANSFERÊNCIA DE CALOR E MASSA II}\\
\vspace{1cm}
\large \uppercase {AVALIAÇÃO FINAL: PROJETO DE DIMENSIONAMENTO DE TROCADORES DE CALOR} \\
 \vspace{1cm}
 \singlespacing
 \hspace{8cm} % posicionando a caixa de texto
 \begin{minipage}{7cm}
	Projeto referente ao tópico Trocadores de Calor, como requisito avaliativo da disciplina Transferência de Calor e Massa II, lecionada na Universidade Federal do Pará. \\
	
	Profa. Dra.: Danielle Regina da Silva Guerra
	\vspace{1cm}
	
	Belém-PA 26 de outubro de 2024
	\vspace{4cm}
\end{minipage}

\onehalfspacing
\begin{center}
	
	EXAMINADOR\\
	\vspace{3cm}
	\rule{10cm}{0.15mm} \\
	Profa. Dra.: Danielle Regina da Silva Guerra\\
	Universidade Federal do Pará - UFPA
\end{center}
\newpage

\begin{center}
    % BLOCO DE FIGURAS
    \thispagestyle{empty}
	\listoffigures
	\newpage
    % SUMARIO
    \thispagestyle{empty}
    \tableofcontents

\end{center}

\newpage
\thispagestyle{empty}
	
\end{center}
    % Resumo, dedicatórias etc (elementos pré extuais)


    % elementos textuais
    % -> introdução: Conteúdo: Apresenta o tema do trabalho, o problema a ser investigado, os objetivos gerais e específicos, a justificativa para a realização do estudo e uma breve descrição da estrutura do documento. Deve contextualizar o leitor sobre a relevância do trabalho.
	\section*{Introdução}

Este solucionário tem como objetivo apresentar a resolução detalhada das questões propostas na lista de exercícios do capítulo 12 do livro "Dinâmica", 12\textordfeminine edição. As questões abrangem os tópicos fundamentais e avançados relacionados à cinemática de mecanismos, com enfoque em problemas práticos e teóricos.\\

A lista é composta por um total de 44 questões, divididas em dois grupos: problemas fundamentais e problemas gerais, exigindo o domínio dos conceitos apresentados no capítulo. As resoluções seguem os procedimentos sugeridos pelo autor do livro, promovendo clareza e rigor matemático para facilitar a compreensão dos conceitos e métodos aplicados.\\

Espera-se que este material contribua para o aprendizado e a consolidação dos conteúdos estudados, além de servir como um guia para a resolução de problemas similares.\\


	% SECÇÃO FUNDAMENTOS
	\section{Questão 12-9}

Nesta questão, analisamos a função da posição \(s(t)\) e determinamos a expressão para a velocidade \(v(t)\) em diferentes intervalos de tempo. Além disso, apresentamos os resultados em forma gráfica.

\subsection*{Função da Posição \(s(t)\)}
A função da posição \(s(t)\) é definida por:
\[
s(t) = 
\begin{cases} 
0.5t^2 & \text{se } t \leq 6, \\
108 & \text{se } t > 6.
\end{cases}
\]

\subsection*{Cálculo da Velocidade \(v(t)\)}
A velocidade \(v(t)\) é obtida pela derivada da posição \(s(t)\) em relação ao tempo \(t\). Para \(t \leq 6\), temos:
\[
s(t) = 0.5t^2 \implies v(t) = \frac{d}{dt}s(t) = t.
\]

Para \(t > 6\), como \(s(t)\) é constante (\(s(t) = 108\)), a velocidade é:
\[
v(t) = 0.
\]

Portanto, a velocidade \(v(t)\) é definida por:
\[
v(t) = 
\begin{cases} 
t & \text{se } t \leq 6, \\
0 & \text{se } t > 6.
\end{cases}
\]

\subsection*{Dados Gerados}
Os dados de tempo (\(t\)), posição (\(s(t)\)), e velocidade (\(v(t)\)) foram gerados e organizados para análise. A tabela a seguir ilustra os valores calculados (valores exemplares):

\begin{table}[H]
    \centering
    \begin{tabular}{|c|c|c|}
        \hline
        \textbf{Tempo (s)} & \textbf{Posição (m)} & \textbf{Velocidade (m/s)} \\
        \hline
        0.0 & 0.0 & 0.0 \\
        1.0 & 0.5 & 1.0 \\
        2.0 & 2.0 & 2.0 \\
        \vdots & \vdots & \vdots \\
        6.0 & 18.0 & 6.0 \\
        7.0 & 108.0 & 0.0 \\
        8.0 & 108.0 & 0.0 \\
        \hline
    \end{tabular}
    \caption{Dados de posição e velocidade em função do tempo.}
\end{table}

\subsection*{Gráfico de Velocidade \(v(t)\)}
A função \(v(t)\) foi representada graficamente. O eixo \(x\) corresponde ao tempo (\(t\)), enquanto o eixo \(y\) corresponde à velocidade (\(v(t)\)). Uma linha vertical foi traçada em \(t = 6\), indicando a mudança no comportamento da função.

\begin{figure}[H]
    \centering
    \begin{tikzpicture}
    \begin{axis}[
        width=12cm, height=8cm,
        xlabel={Tempo (s)},
        ylabel={Velocidade (m/s)},
        grid=major,
        legend pos=north west,
        title={Gráfico Velocidade x Tempo}
    ]
        \addplot[domain=0:6, samples=100, blue, thick] {x};
        \addplot[domain=6:10, samples=100, blue, thick] {0};
        \addplot[dashed, red, thick] coordinates {(6, 0) (6, 6)};
        \legend{$v(t)$, $t = 6$ s (mudança)};
    \end{axis}
    \end{tikzpicture}
    \caption{Gráfico da função velocidade \(v(t)\).}\label{fig:figure}
\end{figure}

\subsection*{Resultados Finais}
\begin{itemize}
    \item Função da posição:
    \[
    s(t) = 
    \begin{cases} 
    0.5t^2 & \text{se } t \leq 6, \\
    108 & \text{se } t > 6.
    \end{cases}
    \]
    \item Função da velocidade:
    \[
    v(t) = 
    \begin{cases} 
    t & \text{se } t \leq 6, \\
    0 & \text{se } t > 6.
    \end{cases}
    \]
\end{itemize}

	\newpage
\section{Questão 12-15}


\begin{figure}[H]
	\centering
	\includegraphics[width=.7\linewidth]{fundamentais/12-15.png}
	\caption{Comando da questão 12-15.}\label{fig:q12-15}
\end{figure}


Nesta questão, determinamos as equações paramétricas das posições \(x(t)\) e \(y(t)\), bem como a relação cartesiana entre as coordenadas \(x\) e \(y\), com base nos componentes da velocidade. A seguir, detalhamos o equacionamento.

\subsection*{Definição das Variáveis e Componentes de Velocidade}
As variáveis e os componentes da velocidade são definidos como:
\[
v_x = 32t, \quad v_y = 8,
\]
onde:
\begin{itemize}
    \item \(t\) representa o tempo;
    \item \(x\) e \(y\) representam as coordenadas no espaço.
\end{itemize}

\subsection*{Integração para Determinar as Posições em Função do Tempo}
A posição na direção \(x\) é obtida pela integração de \(v_x\):
\[
x(t) = \int v_x \, dt = \int 32t \, dt = 16t^2 + C_1.
\]

A posição na direção \(y\) é obtida pela integração de \(v_y\):
\[
y(t) = \int v_y \, dt = \int 8 \, dt = 8t + C_2.
\]

\subsection*{Determinação das Constantes de Integração}
Utilizando as condições iniciais:
\[
x(0) = 0 \quad \text{e} \quad y(0) = 0,
\]
determinamos as constantes \(C_1\) e \(C_2\):
\[
x(0) = 16(0)^2 + C_1 \implies C_1 = 0,
\]
\[
y(0) = 8(0) + C_2 \implies C_2 = 0.
\]

Substituindo as constantes nas equações, obtemos:
\[
x(t) = 16t^2, \quad y(t) = 8t.
\]

\subsection*{Eliminação de \(t\) para Determinar \(y\) em Função de \(x\)}
Da equação de \(x(t)\), resolvemos \(t\) em função de \(x\):
\[
x(t) = 16t^2 \implies t = \sqrt{\frac{x}{16}} = \frac{\sqrt{x}}{4}.
\]

Substituindo \(t\) na equação de \(y(t)\), obtemos:
\[
y = 8t = 8 \cdot \frac{\sqrt{x}}{4} = 2\sqrt{x}.
\]

Portanto, a equação cartesiana entre \(x\) e \(y\) é:
\[
y(x) = 2\sqrt{x}.
\]

\subsection*{Resultados Finais}
\begin{itemize}
    \item Equações paramétricas:
    \[
    x(t) = 16t^2, \quad y(t) = 8t.
    \]
    \item Relação cartesiana entre \(x\) e \(y\):
    \[
    y(x) = 2\sqrt{x}.
    \]
\end{itemize}

	\section{Questão 12-17}

Nesta questão, analisamos o movimento de uma partícula cuja trajetória é definida por uma parábola \(y^2 = 4x\), com a posição em \(x\) dada como função do tempo \(t\). Determinamos as velocidades, acelerações e suas intensidades, bem como os valores numéricos no instante \(t = 0.5 \, \text{s}\).

\subsection*{Equação da Trajetória}
A equação da trajetória da partícula é definida como:
\[
y^2 = 4x,
\]
onde a posição \(x\) é dada por:
\[
x(t) = 4t^4.
\]

\subsection*{Cálculo das Derivadas para \(x\)}
A velocidade na direção \(x\) é obtida pela derivada de \(x(t)\) em relação ao tempo \(t\):
\[
v_x = \frac{dx}{dt} = \frac{d}{dt}\left(4t^4\right) = 16t^3.
\]

A aceleração na direção \(x\) é a derivada de \(v_x\):
\[
a_x = \frac{dv_x}{dt} = \frac{d}{dt}\left(16t^3\right) = 48t^2.
\]

\subsection*{Substituição de \(x\) na Equação da Trajetória}
Substituímos \(x(t)\) na equação da trajetória para encontrar \(y(t)\):
\[
y^2 = 4x \implies y^2 = 4(4t^4) \implies y = 4t^2.
\]

\subsection*{Cálculo das Derivadas para \(y\)}
A velocidade na direção \(y\) é:
\[
v_y = \frac{dy}{dt} = \frac{d}{dt}\left(4t^2\right) = 8t.
\]

A aceleração na direção \(y\) é:
\[
a_y = \frac{dv_y}{dt} = \frac{d}{dt}\left(8t\right) = 8.
\]

\subsection*{Intensidade da Velocidade}
A intensidade da velocidade é dada por:
\[
|\vec{v}| = \sqrt{v_x^2 + v_y^2}.
\]

Substituindo \(v_x\) e \(v_y\):
\[
|\vec{v}| = \sqrt{(16t^3)^2 + (8t)^2} = \sqrt{256t^6 + 64t^2}.
\]

\subsection*{Intensidade da Aceleração}
A intensidade da aceleração é dada por:
\[
|\vec{a}| = \sqrt{a_x^2 + a_y^2}.
\]

Substituindo \(a_x\) e \(a_y\):
\[
|\vec{a}| = \sqrt{(48t^2)^2 + (8)^2} = \sqrt{2304t^4 + 64}.
\]

\subsection*{Cálculos no Instante \(t = 0.5 \, \text{s}\)}
Substituímos \(t = 0.5 \, \text{s}\) nas equações para obter os valores numéricos:
\begin{itemize}
    \item Velocidade em \(x\): 
    \[
    v_x = 16t^3 \implies v_x = 16(0.5)^3 = 2.0 \, \text{m/s}.
    \]
    \item Velocidade em \(y\): 
    \[
    v_y = 8t \implies v_y = 8(0.5) = 4.0 \, \text{m/s}.
    \]
    \item Intensidade da velocidade:
    \[
    |\vec{v}| = \sqrt{256(0.5)^6 + 64(0.5)^2} \implies |\vec{v}| \approx 4.47 \, \text{m/s}.
    \]
    \item Intensidade da aceleração:
    \[
    |\vec{a}| = \sqrt{2304(0.5)^4 + 64} \implies |\vec{a}| \approx 34.06 \, \text{m/s}^2.
    \]
\end{itemize}

\subsection*{Resultados Finais}
\begin{itemize}
    \item Equações paramétricas:
    \[
    x(t) = 4t^4, \quad y(t) = 4t^2.
    \]
    \item Velocidade:
    \[
    v_x = 16t^3, \quad v_y = 8t.
    \]
    \item Aceleração:
    \[
    a_x = 48t^2, \quad a_y = 8.
    \]
    \item Intensidades:
    \[
    |\vec{v}| = \sqrt{256t^6 + 64t^2}, \quad |\vec{a}| = \sqrt{2304t^4 + 64}.
    \]
    \item Valores no instante \(t = 0.5 \, \text{s}\):
    \begin{itemize}
        \item \(v_x = 2.0 \, \text{m/s}\),
        \item \(v_y = 4.0 \, \text{m/s}\),
        \item \(|\vec{v}| \approx 4.47 \, \text{m/s}\),
        \item \(|\vec{a}| \approx 34.06 \, \text{m/s}^2\).
    \end{itemize}
\end{itemize}

	\newpage
\section{Questão 12-20}

\begin{figure}[H]
	\centering
	\includegraphics[width=0.7\linewidth]{fundamentais/12-20.png}
	\caption{Comando da questão 12-20.}\label{fig:q12-20}
\end{figure}

Nesta questão, analisamos a posição, velocidade e aceleração de uma partícula cujo movimento é descrito por uma função vetorial em um espaço tridimensional. Determinamos as expressões para a velocidade e aceleração vetoriais e avaliamos seus valores numéricos no instante \(t = 2 \, \text{s}\).

\subsection*{Função Vetorial da Posição}
A posição da partícula é descrita pela função vetorial:
\[
\vec{r}(t) = 2 \sin(2t) \, \hat{i} + 2 \cos(t) \, \hat{j} - 2t^2 \, \hat{k},
\]
onde:
\begin{itemize}
    \item \(\hat{i}, \hat{j}, \hat{k}\) são os vetores unitários nas direções \(x\), \(y\) e \(z\), respectivamente;
    \item \(t\) é o tempo.
\end{itemize}

\subsection*{Velocidade Vetorial}
A velocidade da partícula é obtida pela derivada de \(\vec{r}(t)\) em relação ao tempo:
\[
\vec{v}(t) = \frac{d\vec{r}(t)}{dt}.
\]

Calculando cada componente:
\[
\vec{v}(t) = 4 \cos(2t) \, \hat{i} - 2 \sin(t) \, \hat{j} - 4t \, \hat{k}.
\]

\subsection*{Aceleração Vetorial}
A aceleração da partícula é obtida pela derivada de \(\vec{v}(t)\) em relação ao tempo:
\[
\vec{a}(t) = \frac{d\vec{v}(t)}{dt}.
\]

Calculando cada componente:
\[
\vec{a}(t) = -8 \sin(2t) \, \hat{i} - 2 \cos(t) \, \hat{j} - 4 \, \hat{k}.
\]

\subsection*{Valores Numéricos no Instante \(t = 2 \, \text{s}\)}
Substituímos \(t = 2 \, \text{s}\) nas expressões de \(\vec{v}(t)\) e \(\vec{a}(t)\) para calcular seus valores numéricos:

\[
\vec{v}(2) = 4 \cos(4) \, \hat{i} - 2 \sin(2) \, \hat{j} - 8 \, \hat{k}.
\]

\[
\vec{a}(2) = -8 \sin(4) \, \hat{i} - 2 \cos(2) \, \hat{j} - 4 \, \hat{k}.
\]

\subsection*{Resultados Finais}
\begin{itemize}
    \item Velocidade vetorial:
    \[
    \vec{v}(t) = 4 \cos(2t) \, \hat{i} - 2 \sin(t) \, \hat{j} - 4t \, \hat{k}.
    \]
    Valor no instante \(t = 2 \, \text{s}\):
    \[
    \vec{v}(2) = 4 \cos(4) \, \hat{i} - 2 \sin(2) \, \hat{j} - 8 \, \hat{k} = -2.614 \,\hat{i} + 1.8185 \,\hat{j} - 8\,\hat{k}
    \]

    \item Aceleração vetorial:
    \[
    \vec{a}(t) = -8 \sin(2t) \, \hat{i} - 2 \cos(t) \, \hat{j} - 4 \, \hat{k}.
    \]
    Valor no instante \(t = 2 \, \text{s}\):
    \[
    \vec{a}(2) = -8 \sin(4) \, \hat{i} - 2 \cos(2) \, \hat{j} - 4 \, \hat{k} = 6.0544 \,\hat{i} + 0.8323 \,\hat{j} - 4\,\hat{k}
    \]
\end{itemize}

	\newpage
\section{Questão 12-22}

\begin{figure}[H]
	\centering
	\includegraphics[width=.7\linewidth]{fundamentais/12-22.png}
	\caption{Comando da questão 12-22.}\label{fig?:12-22}
\end{figure}

Nesta questão, analisamos o movimento de um projétil lançado obliquamente com velocidade inicial \(v_A\) e ângulo de lançamento \(\alpha = 30^\circ\). Determinamos o tempo total de voo, o alcance horizontal (\(R\)) e a velocidade escalar no impacto. Também substituímos valores numéricos para ilustrar os resultados.

\subsection*{Componentes da Velocidade Inicial}
As componentes da velocidade inicial são:
\[
v_{Ax} = v_A \cos(\alpha),
\]
\[
v_{Ay} = v_A \sin(\alpha),
\]
onde:
\begin{itemize}
    \item \(v_{Ax}\): Componente horizontal da velocidade;
    \item \(v_{Ay}\): Componente vertical da velocidade.
\end{itemize}

\subsection*{Equações do Movimento}
A aceleração pode ser determinada por
\[
a(t) = \frac{dv}{dt}
\]

E a velocidade é determinada por:
\[
v(t) = \frac{ds}{dt}
\]
Manipulando as equações e combinando-as através de \(dt\), temos:

\[
a\,ds = v\,dv 
\]

Esta equação será a base das análises daqui em diante.

\subsection*{Tempo Total de Voo}

Consideramos que a aceleração horizontal \(a_x = 0\) e a vertical como \(a_y = -g\), o que permite que o alcance \(R\) seja determinado pelo tempo de voo.

O tempo total de voo ocorre quando \(y = 0\). Resolvemos a equação \(y(t) = 0\):
\[
v_{Ay} \cdot t - \frac{1}{2} g t^2 = 0.
\]

Fatorando \(t\), temos:
\[
t \left( v_{Ay} - \frac{1}{2} g t \right) = 0.
\]

A solução positiva é:
\[
t_{\text{total}} = \frac{2 v_{Ay}}{g}.
\]

\subsection*{Alcance Horizontal (\(R\))}
Substituímos \(t_{\text{total}}\) na equação do movimento horizontal para determinar o alcance:
\[
R = x(t_{\text{total}}) = v_{Ax} \cdot t_{\text{total}}.
\]

Substituindo \(t_{\text{total}} = \frac{2 v_{Ay}}{g}\):
\[
R = v_{Ax} \cdot \frac{2 v_{Ay}}{g}.
\]

Usando as expressões para \(v_{Ax}\) e \(v_{Ay}\):
\[
R = \frac{2 v_A^2 \sin(\alpha) \cos(\alpha)}{g}.
\]

Simplificando com a identidade trigonométrica \(\sin(2\alpha) = 2 \sin(\alpha) \cos(\alpha)\):
\[
R = \frac{v_A^2 \sin(2\alpha)}{g}.
\]

\subsection*{Velocidade Escalar no Impacto}
A componente horizontal da velocidade no impacto permanece constante:
\[
v_{x,\text{final}} = v_{Ax}.
\]

A componente vertical no impacto é:
\[
v_{y,\text{final}} = v_{Ay} - g \cdot t_{\text{total}}.
\]

Substituindo \(t_{\text{total}} = \frac{2 v_{Ay}}{g}\):
\[
v_{y,\text{final}} = v_{Ay} - g \cdot \frac{2 v_{Ay}}{g} = -v_{Ay}.
\]

A velocidade escalar no impacto é dada por:
\[
v_{\text{final}} = \sqrt{v_{x,\text{final}}^2 + v_{y,\text{final}}^2}.
\]

Substituindo os valores de \(v_{x,\text{final}}\) e \(v_{y,\text{final}}\):
\[
v_{\text{final}} = \sqrt{v_{Ax}^2 + (-v_{Ay})^2} = \sqrt{v_A^2}. = v_A
\]

\subsection*{Cálculos Numéricos}
Substituímos os seguintes valores:
\[
v_A = 10 \, \text{m/s}, \quad \alpha = 30^\circ = \frac{\pi}{6}, \quad g = 9.81 \, \text{m/s}^2.
\]

O alcance horizontal é:
\[
R = \frac{10^2 \sin(2 \cdot 30^\circ)}{9.81} = \frac{100 \cdot 0.866}{9.81} \approx 8.827 \, \text{m}.
\]

A velocidade escalar no impacto é:
\[
v_{\text{final}} = \sqrt{10^2} = 10 \, \text{m/s}.
\]

\subsection*{Resultados Finais}
\begin{itemize}
    \item Tempo total de voo:
    \[
    t_{\text{total}} = \frac{2 v_A \sin(\alpha)}{g}.
    \]
    \item Alcance horizontal:
    \[
    R = \frac{v_A^2 \sin(2\alpha)}{g} \approx 8.81 \, \text{m}.
    \]
    \item Velocidade escalar no impacto:
    \[
    v_{\text{final}} = v_A = 10 \, \text{m/s}.
    \]
\end{itemize}

	\newpage
\section{Questão 12-23}

\begin{figure}[H]
	\centering
	\includegraphics[width=.7\linewidth]{fundamentais/12-23.png}
	\caption{Comando da questão 12-23.}\label{fig:12-23}
\end{figure}

Nesta questão, analisamos o movimento de um projétil lançado de uma altura inicial \(y_0 = 1.5 \, \text{m}\) com um ângulo de lançamento de \(30^\circ\). O projétil percorre uma distância horizontal de \(x = 10 \, \text{m}\) e atinge uma altura final de \(y_f = 3 \, \text{m}\). Nosso objetivo é determinar a velocidade inicial \(v_A\) necessária para satisfazer essas condições.

\subsection*{Equações do Movimento}
As equações do movimento horizontal e vertical são:
\[
x = v_A \cdot \cos(\theta) \cdot t,
\]
\[
y_f = y_0 + v_A \cdot \sin(\theta) \cdot t - \frac{1}{2} g \cdot t^2,
\]
onde:
\begin{itemize}
    \item \(x = 10 \, \text{m}\): Distância horizontal;
    \item \(y_0 = 1.5 \, \text{m}\): Altura inicial;
    \item \(y_f = 3 \, \text{m}\): Altura final;
    \item \(g = 9.81 \, \text{m/s}^2\): Aceleração gravitacional;
    \item \(\theta = 30^\circ = \frac{\pi}{6}\): Ângulo de lançamento.
\end{itemize}

\subsection*{Movimento Horizontal}
Do movimento horizontal, temos:
\[
x = v_A \cdot \cos(\theta) \cdot t.
\]

Resolvendo para o tempo \(t\):
\[
t = \frac{x}{v_A \cdot \cos(\theta)}.
\]

\subsection*{Movimento Vertical}
Substituímos \(t = \frac{x}{v_A \cdot \cos(\theta)}\) na equação do movimento vertical:
\[
y_f = y_0 + v_A \cdot \sin(\theta) \cdot \frac{x}{v_A \cdot \cos(\theta)} - \frac{1}{2} g \cdot \left(\frac{x}{v_A \cdot \cos(\theta)}\right)^2.
\]

Simplificando:
\[
y_f = y_0 + x \cdot \tan(\theta) - \frac{g \cdot x^2}{2 \cdot v_A^2 \cdot \cos^2(\theta)}.
\]

Substituímos \(y_0 = 1.5 \, \text{m}\), \(y_f = 3 \, \text{m}\), \(x = 10 \, \text{m}\), \(g = 9.81 \, \text{m/s}^2\), e \(\cos(30^\circ) = \sqrt{3}/2\), \(\tan(30^\circ) = 1/\sqrt{3}\):
\[
3 = 1.5 + 10 \cdot \frac{1}{\sqrt{3}} - \frac{9.81 \cdot 10^2}{2 \cdot v_A^2 \cdot \left(\frac{\sqrt{3}}{2}\right)^2}.
\]

Simplificando:
\[
3 = 1.5 + \frac{10}{\sqrt{3}} - \frac{9.81 \cdot 100}{v_A^2 \cdot \frac{3}{4}}.
\]

\[
3 = 1.5 + \frac{10}{\sqrt{3}} - \frac{1308}{v_A^2}.
\]

\subsection*{Resolução para \(v_A\)}
Reorganizamos a equação para resolver \(v_A\):
\[
v_A^2 = \frac{1308}{3 - 1.5 - \frac{10}{\sqrt{3}}}.
\]

Calculando:
\[
v_A \approx 12.37 \, \text{m/s}.
\]

\subsection*{Resultado Final}
A velocidade inicial necessária para que o projétil atinja a altura final \(y_f = 3 \, \text{m}\) após percorrer \(x = 10 \, \text{m}\) é:
\[
v_A \approx 12.37 \, \text{m/s}.
\]

	\section{Questão 12-27}

Nesta questão, analisamos o movimento de uma partícula em uma trajetória circular de raio \(r = 40 \, \text{m}\), cuja velocidade escalar é dada por \(v(t) = 0.0625 \cdot t^2\) (em m/s). Calculamos as acelerações tangencial, centrípeta e total (resultante) e avaliamos seus valores no instante \(t = 10 \, \text{s}\).

\subsection*{Aceleração Tangencial}
A aceleração tangencial é obtida como a derivada da velocidade escalar em relação ao tempo:
\[
a_t = \frac{dv}{dt}.
\]

Derivando \(v(t) = 0.0625 \cdot t^2\):
\[
a_t = \frac{d}{dt}\left(0.0625 \cdot t^2\right) = 0.125 \cdot t.
\]

\subsection*{Aceleração Centrípeta}
A aceleração centrípeta é dada por:
\[
a_c = \frac{v^2}{r}.
\]

Substituímos \(v(t) = 0.0625 \cdot t^2\) e \(r = 40 \, \text{m}\):
\[
a_c = \frac{\left(0.0625 \cdot t^2\right)^2}{40} = \frac{0.00390625 \cdot t^4}{40} = 0.00009765625 \cdot t^4.
\]

\subsection*{Aceleração Total (Resultante)}
A aceleração total é a soma vetorial das componentes tangencial e centrípeta:
\[
a_{\text{total}} = \sqrt{a_t^2 + a_c^2}.
\]

Substituímos \(a_t = 0.125 \cdot t\) e \(a_c = 0.00009765625 \cdot t^4\):
\[
a_{\text{total}} = \sqrt{\left(0.125 \cdot t\right)^2 + \left(0.00009765625 \cdot t^4\right)^2}.
\]

\subsection*{Cálculos no Instante \(t = 10 \, \text{s}\)}
Substituímos \(t = 10 \, \text{s}\) nas expressões para calcular os valores numéricos:
\begin{itemize}
    \item Aceleração tangencial:
    \[
    a_t = 0.125 \cdot 10 = 1.25 \, \text{m/s}^2.
    \]
    \item Aceleração centrípeta:
    \[
    a_c = 0.00009765625 \cdot 10^4 = 9.765625 \, \text{m/s}^2.
    \]
    \item Aceleração total:
    \[
    a_{\text{total}} = \sqrt{1.25^2 + 9.765625^2} \approx 9.844 \, \text{m/s}^2.
    \]
\end{itemize}

\subsection*{Resultados Finais}
\begin{itemize}
    \item Aceleração tangencial:
    \[
    a_t = 0.125 \cdot t \quad \text{(em \(t = 10 \, \text{s}\): \(a_t = 1.25 \, \text{m/s}^2\))}.
    \]
    \item Aceleração centrípeta:
    \[
    a_c = 0.00009765625 \cdot t^4 \quad \text{(em \(t = 10 \, \text{s}\): \(a_c = 9.765625 \, \text{m/s}^2\))}.
    \]
    \item Aceleração total:
    \[
    a_{\text{total}} = \sqrt{\left(0.125 \cdot t\right)^2 + \left(0.00009765625 \cdot t^4\right)^2} \quad \text{(em \(t = 10 \, \text{s}\): \(a_{\text{total}} \approx 9.844 \, \text{m/s}^2\))}.
    \]
\end{itemize}

	\newpage
\section{Questão 12-30}

\begin{figure}[H]
	\centering
	\includegraphics[width=0.7\linewidth]{fundamentais/12-30.png}
	\caption{Comando da questão 12-30}\label{fig:q12-30}
\end{figure}

Nesta questão, analisamos o movimento de uma partícula cuja trajetória é descrita por \(y = \frac{1}{8}x^2\). Sabendo que a velocidade escalar é constante (\(v = 6 \, \text{m/s}\)) e que a aceleração tangencial é \(a_t = 1.8 \, \text{m/s}^2\), determinamos o ângulo de inclinação da trajetória (\(\theta\)), a aceleração normal (\(a_n\)) e a aceleração total (\(a_{\text{total}}\)) no ponto \(x = 3 \, \text{m}\).

\subsection*{Equação da Trajetória}
A equação da trajetória é dada por:
\[
y = \frac{1}{8}x^2.
\]

\subsection*{Derivada da Trajetória e Ângulo de Inclinação}
A inclinação da trajetória é obtida pela derivada de \(y\) em relação a \(x\):
\[
\frac{dy}{dx} = \frac{1}{4}x.
\]

O ângulo de inclinação \(\theta\) é dado por:
\[
\theta = \arctan\left(\frac{dy}{dx}\right).
\]

Substituindo \(x = 3 \, \text{m}\):
\[
\theta = \arctan\left(\frac{1}{4} \cdot 3\right) = \arctan\left(\frac{3}{4}\right).
\]

Convertendo para graus:
\[
\theta \approx 36.87^\circ.
\]

\subsection*{Segunda derivada}
A derivada de segunda ordem, ou segunda derivada de \(y\), é:

\[\frac{d^2y}{dx^2} = \frac{d^2}{dx^2}\left(\frac{x}{4}\right) = \frac{1}{4} \]

\subsection*{Aceleração Normal (\(a_n\))}
A aceleração normal é calculada como:

\[
a_n = \frac{v^2}{\rho}
\]

onde:

\[
\rho = \frac{\left[1 + \left(\frac{dy}{dx}\right)^2 \right]^{(3/2)}}{ 
\left|\frac{d^2y}{dx^2} \right|}
\]


Substituímos \(v = 6 \, \text{m/s}\), \(\frac{dy}{dx} = \frac{1}{4}x\) e \(\frac{d^2y}{dx^2} = \frac{1}{4}\):

\[
a_n = \frac{6^2 \cdot \left|\frac{1}{4}\right|}{\sqrt{\left(1 + \left(\frac{1}{4}x\right)^2\right)^3}}.
\]

Para \(x = 3 \, \text{m}\):
\[
a_n = \frac{6^2 \cdot \left|\frac{1}{4}\right|}{\sqrt{\left(1 + \left(\frac{1}{4} \cdot 3\right)^2\right)^3}} = \frac{9}{\left(\sqrt{1 + \left(\frac{3}{4}\right)^2}\right)^3}.
\]

Simplificando:
\[
a_n \approx 4.608 \, \text{m/s}^2.
\]

\subsection*{Aceleração Total (\(a_{\text{total}}\))}
A aceleração total é a soma vetorial das componentes tangencial e normal:
\[
a_{\text{total}} = \sqrt{a_t^2 + a_n^2}.
\]

Substituímos \(a_t = 1.8 \, \text{m/s}^2\) e \(a_n \approx 8.57 \, \text{m/s}^2\):
\[
a_{\text{total}} = \sqrt{1.8^2 + 4.61^2}.
\]

Simplificando:
\[
a_{\text{total}} \approx 4.948 \, \text{m/s}^2.
\]

\subsection*{Resultados Finais}
\begin{itemize}
    \item Ângulo de inclinação:
    \[
    \theta \approx 36.87^\circ \quad \text{(em \(x = 3 \, \text{m}\))}.
    \]
    \item Aceleração normal:
    \[
    a_n \approx 4.608 \, \text{m/s}^2 \quad \text{(em \(x = 3 \, \text{m}\))}.
    \]
    \item Aceleração total:
    \[
    a_{\text{total}} \approx 4.948 \, \text{m/s}^2 \quad \text{(em \(x = 3 \, \text{m}\))}.
    \]
\end{itemize}

	\section{Questão:12-31}

Nesta questão, analisamos o movimento de uma partícula ao longo de uma curva circular com raio \(r = 300 \, \text{m}\). A aceleração tangencial da partícula é variável e descrita pela equação \(a_t = -0.001 \cdot s\), onde \(s\) é a posição ao longo do arco em metros. Sabemos que a velocidade da partícula no ponto \(A\) (\(s = 0\)) é \(v_A = 25 \, \text{m/s}\). Nosso objetivo é determinar a velocidade da partícula no ponto \(B\) (\(s = r = 300 \, \text{m}\)).

\subsection*{Equação do Movimento}
A equação do movimento é dada por:
\[
a_t = v \cdot \frac{dv}{ds},
\]
onde:
\begin{itemize}
    \item \(a_t = -0.001 \cdot s\): Aceleração tangencial variável;
    \item \(v\): Velocidade escalar da partícula;
    \item \(s\): Posição ao longo do arco.
\end{itemize}

Substituímos \(a_t\) na equação:
\[
-0.001 \cdot s = v \cdot \frac{dv}{ds}.
\]

Reorganizando:
\[
v \cdot dv = -0.001 \cdot s \cdot ds.
\]

\subsection*{Integração}
Integramos ambos os lados para determinar \(v\) em função de \(s\). No ponto \(A\), temos \(v = v_A = 25 \, \text{m/s}\) quando \(s = 0\):
\[
\int_{v_A}^{v} v \, dv = \int_{0}^{s} -0.001 \cdot s \, ds.
\]

Resolvendo a integral do lado esquerdo:
\[
\frac{v^2}{2} \bigg|_{v_A}^{v} = -0.001 \cdot \frac{s^2}{2} \bigg|_{0}^{s}.
\]

Substituímos os limites:
\[
\frac{v^2}{2} - \frac{v_A^2}{2} = -0.001 \cdot \frac{s^2}{2}.
\]

Reorganizando para \(v^2\):
\[
v^2 = v_A^2 - 0.001 \cdot s^2.
\]

\subsection*{Velocidade no Ponto \(B\)}
No ponto \(B\), \(s = r = 300 \, \text{m}\). Substituímos \(s = 300 \, \text{m}\) e \(v_A = 25 \, \text{m/s}\) na equação:
\[
v^2 = 25^2 - 0.001 \cdot (300)^2.
\]

Calculando:
\[
v^2 = 625 - 0.001 \cdot 90000.
\]

\[
v^2 = 625 - 90 = 535.
\]

A velocidade no ponto \(B\) é:
\[
v = \sqrt{535} \approx 23.13 \, \text{m/s}.
\]

\subsection*{Resultado Final}
A velocidade da partícula no ponto \(B\) (\(s = r = 300 \, \text{m}\)) é:
\[
v \approx 23.13 \, \text{m/s}.
\]

	\newpage
\section{Questão 12-33}

\begin{figure}[H]
	\centering
	\includegraphics[width=.7\linewidth]{fundamentais/12-33.png}
	\caption{Comando da questão 12-33}\label{fig:12-33}
\end{figure}

Nesta questão, analisamos o movimento de uma partícula em uma trajetória circular com raio \(r = 120 \, \text{m}\) e velocidade escalar \(v = 16.5 \, \text{m/s}\). Determinamos a velocidade angular \(\dot{\theta}\) da partícula.

\subsection*{Cálculo da Velocidade Angular}
A relação entre a velocidade angular \(\dot{\theta}\) e a velocidade escalar \(v\) em uma trajetória circular é dada por:
\[
\dot{\theta} = \frac{v}{r},
\]
onde:
\begin{itemize}
    \item \(\dot{\theta}\): Velocidade angular (em rad/s);
    \item \(v\): Velocidade escalar (em m/s);
    \item \(r\): Raio da trajetória circular (em m).
\end{itemize}

\subsection*{Substituição dos Valores Numéricos}
Substituímos os valores \(v = 16.5 \, \text{m/s}\) e \(r = 120 \, \text{m}\) na equação:
\[
\dot{\theta} = \frac{16.5}{120}.
\]

Simplificando:
\[
\dot{\theta} \approx 0.138 \, \text{rad/s}.
\]

\subsection*{Resultado Final}
A velocidade angular da partícula é:
\[
\dot{\theta} \approx 0.138 \, \text{rad/s}.
\]

	\newpage
\section{Questão 12-34}

\begin{figure}[H]
	\centering
	\includegraphics[width=.7\linewidth]{fundamentais/12-34.png}
	\caption{Comando da questão 12-34.}\label{fig:12-34}
\end{figure}


Nesta questão, analisamos o movimento de uma partícula em coordenadas polares, onde a posição radial e a posição angular variam com o tempo. A posição radial é dada por \(r(t) = 0.1 \cdot t^3\), e a posição angular é \( \theta(t) = 4 \cdot t^{3/2}\). Calculamos as velocidades, acelerações e suas intensidades no instante \(t = 1.5 \, \text{s}\).

\subsection*{Velocidade Radial e Angular}
A velocidade radial é a derivada de \(r(t)\) em relação ao tempo:
\[
\frac{dr}{dt} = \frac{d}{dt}\left(0.1 \cdot t^3\right) = 0.3 \cdot t^2.
\]

E a segunda derivada é:
\[
\frac{d^2r}{dt^2} = \frac{d^2\left(0.3t^2\right)}{dt^2} = 0.6t
\]

A velocidade angular é a derivada de \(\theta(t)\) em relação ao tempo:
\[
\frac{d\theta}{dt} = \frac{d}{dt}\left(4 \cdot t^{3/2}\right) = 6 \cdot t^{1/2}.
\]

E sua respectiva segunda derivada é:

\[
\frac{d^2\theta}{dt^2} = \frac{d^2\left(6\cdot t^{1/2}\right)}{dt^2} = 3\cdot t^{-1/2}
\]

\subsection*{Velocidade Tangencial e Intensidade da Velocidade Total}
A velocidade tangencial é dada por:
\[
v_{\text{tangencial}} = r \cdot \frac{d\theta}{dt}.
\]

Substituímos \(r(t) = 0.1 \cdot t^3\) e \(\frac{d\theta}{dt} = 6 \cdot t^{1/2}\):
\[
v_{\text{tangencial}} = \left(0.1 \cdot t^3\right) \cdot \left(6 \cdot t^{1/2}\right) = 0.6 \cdot t^{7/2}.
\]

A intensidade da velocidade total é:
\[
v_{\text{total}} = \sqrt{\left(\frac{dr}{dt}\right)^2 + v_{\text{tangencial}}^2}.
\]

Substituímos \(\frac{dr}{dt} = 0.3 \cdot t^2\) e \(v_{\text{tangencial}} = 0.6 \cdot t^{7/2}\):
\[
v_{\text{total}} = \sqrt{\left(0.3 \cdot t^2\right)^2 + \left(0.6 \cdot t^{7/2}\right)^2}.
\]

\subsection*{Acelerações Radial e Tangencial}
A aceleração radial (centrípeta) é dada por:

\[
a_{\text{radial}} = \frac{d^2r}{dt^2} - r\cdot \left(\frac{d\theta}{dt}\right)^2
\]


Substituímos \(r(t) = 0.1 \cdot t^3\), \(\frac{d^2r}{dt^2} = 0.6t\) e \(\frac{d\theta}{dt} = 6 \cdot t^{1/2}\):
\[
a_{\text{radial}} = \left(0.6t\right) - 0.1t^3 \cdot (6 t^{1/2})^2 = 0.6t - 3.6t^4.
\]


A aceleração tangencial em relação ao tempo vale:
\[
a_{\text{tangencial}} = r \frac{d^2\theta}{dt^2} + 2\frac{dr}{dt}\cdot \frac{d\theta}{dt}
\]

\[
a_{\text{tangencial}} = 0.1t^3 \cdot 3t^{-1/2} + 2\cdot 0.3t^2\cdot 6t^{1/2} = 0.3t^{5/2} + 3.6 t^{5/2} = 3.9 t^{5/2}
\]

A intensidade da aceleração total é:
\[
a_{\text{total}} = \sqrt{a_{\text{radial}}^2 + a_{\text{tangencial}}^2}.
\]

Substituímos \(a_{\text{radial}} = 0.6t - 3.6 \cdot t^4\) e \(a_{\text{tangencial}} = 3.9 \cdot t^{5/2}\):
\[
a_{\text{total}} = \sqrt{\left(0.6t - 3.6 \cdot t^4\right)^2 + \left(3.9 \cdot t^{5/2}\right)^2}.
\]

\subsection*{Cálculos no Instante \(t = 1.5 \, \text{s}\)}
Substituímos \(t = 1.5 \, \text{s}\) nas expressões:
\begin{itemize}
    \item Velocidade total:
    \[
    v_{\text{total}} = \sqrt{\left(0.3 \cdot 1.5^2\right)^2 + \left(0.6 \cdot 1.5^{7/2}\right)^2} \approx 2.57032 \, \text{m/s}.
    \]
    \item Aceleração total:
    \[
    a_{\text{total}} = \sqrt{\left(3.6 \cdot 1.5^4\right)^2 + \left(2.1 \cdot 1.5^{5/2}\right)^2} \approx 20.3877 \, \text{m/s}^2.
    \]
\end{itemize}
	\section{Questão 12-39}

Nesta questão, analisamos um sistema de polias em que a velocidade de um ponto \(A\) é relacionada à velocidade de um bloco \(D\) devido à configuração do sistema de cordas. Determinamos a velocidade de \(D\) (\(v_D\)) quando a velocidade de \(A\) (\(v_A\)) é \(3 \, \text{m/s}\).

\subsection*{Relação entre as Velocidades}
No sistema de polias, a velocidade de \(A\) é o dobro da velocidade de \(D\), pois a corda se divide em duas partes móveis conectadas ao bloco \(D\). Assim, temos a relação:
\[
v_A = 2 \cdot v_D,
\]
onde:
\begin{itemize}
    \item \(v_A\): Velocidade no ponto \(A\);
    \item \(v_D\): Velocidade no bloco \(D\).
\end{itemize}

\subsection*{Cálculo de \(v_D\)}
Substituímos \(v_A = 3 \, \text{m/s}\) na equação:
\[
3 = 2 \cdot v_D.
\]

Resolvendo para \(v_D\):
\[
v_D = \frac{3}{2} = 1.5 \, \text{m/s}.
\]

\subsection*{Resultado Final}
A velocidade do bloco \(D\) é:
\[
v_D = 1.5 \, \text{m/s}.
\]

	\section{Questão 12-40}

Nesta questão, analisamos um sistema de polias em que a velocidade no ponto \(B\) é relacionada à velocidade do bloco \(A\) devido à configuração do sistema de cordas. Determinamos a velocidade de \(A\) (\(v_A\)) quando a velocidade de \(B\) (\(v_B\)) é \(6 \, \text{m/s}\).

\subsection*{Relação entre as Velocidades}
No sistema de polias, a velocidade no ponto \(B\) é o dobro da velocidade do bloco \(A\), pois a corda se divide em duas partes móveis conectadas ao bloco \(A\). Assim, temos a relação:
\[
v_B = 2 \cdot v_A,
\]
onde:
\begin{itemize}
    \item \(v_B\): Velocidade no ponto \(B\);
    \item \(v_A\): Velocidade no bloco \(A\).
\end{itemize}

\subsection*{Cálculo de \(v_A\)}
Substituímos \(v_B = 6 \, \text{m/s}\) na equação:
\[
6 = 2 \cdot v_A.
\]

Resolvendo para \(v_A\):
\[
v_A = \frac{6}{2} = 3 \, \text{m/s}.
\]

\subsection*{Resultado Final}
A velocidade do bloco \(A\) é:
\[
v_A = 3 \, \text{m/s}.
\]

	\section{Questão 12-47}

Nesta questão, analisamos o movimento de dois barcos, \(A\) e \(B\), que se movem em direções diferentes. O barco \(A\) se move com uma velocidade escalar \(v_A = 15 \, \text{m/s}\) a um ângulo de \(30^\circ\) em relação ao eixo \(x\), enquanto o barco \(B\) se move com uma velocidade escalar \(v_B = 10 \, \text{m/s}\) na direção do eixo \(y\). Determinamos a distância entre os barcos no instante \(t = 4 \, \text{s}\).

\subsection*{Posições dos Barcos}
A posição do barco \(A\) é dada por suas componentes \(x_A\) e \(y_A\):
\[
x_A = v_A \cdot t \cdot \cos(\theta),
\]
\[
y_A = v_A \cdot t \cdot \sin(\theta),
\]
onde:
\begin{itemize}
    \item \(v_A = 15 \, \text{m/s}\) é a velocidade escalar do barco \(A\);
    \item \(\theta = 30^\circ = \frac{\pi}{6}\) é o ângulo do movimento do barco \(A\).
\end{itemize}

A posição do barco \(B\) é:
\[
x_B = 0, \quad y_B = v_B \cdot t,
\]
onde \(v_B = 10 \, \text{m/s}\) é a velocidade escalar do barco \(B\).

\subsection*{Distância entre os Barcos}
A distância entre os barcos é dada por:
\[
d(t) = \sqrt{(x_A - x_B)^2 + (y_A - y_B)^2}.
\]

Substituímos as expressões para \(x_A\), \(x_B\), \(y_A\) e \(y_B\):
\[
d(t) = \sqrt{\left(v_A \cdot t \cdot \cos(\theta) - 0\right)^2 + \left(v_A \cdot t \cdot \sin(\theta) - v_B \cdot t\right)^2}.
\]

Simplificando:
\[
d(t) = \sqrt{\left(15 \cdot t \cdot \cos\left(\frac{\pi}{6}\right)\right)^2 + \left(15 \cdot t \cdot \sin\left(\frac{\pi}{6}\right) - 10 \cdot t\right)^2}.
\]

\subsection*{Cálculo para \(t = 4 \, \text{s}\)}
Substituímos \(t = 4 \, \text{s}\) na expressão:
\[
d(4) = \sqrt{\left(15 \cdot 4 \cdot \cos\left(\frac{\pi}{6}\right)\right)^2 + \left(15 \cdot 4 \cdot \sin\left(\frac{\pi}{6}\right) - 10 \cdot 4\right)^2}.
\]

Calculando:
\[
\cos\left(\frac{\pi}{6}\right) = \frac{\sqrt{3}}{2}, \quad \sin\left(\frac{\pi}{6}\right) = \frac{1}{2}.
\]

Substituímos:
\[
d(4) = \sqrt{\left(15 \cdot 4 \cdot \frac{\sqrt{3}}{2}\right)^2 + \left(15 \cdot 4 \cdot \frac{1}{2} - 10 \cdot 4\right)^2}.
\]

Simplificando:
\[
d(4) = \sqrt{\left(60 \cdot \frac{\sqrt{3}}{2}\right)^2 + \left(30 - 40\right)^2},
\]
\[
d(4) = \sqrt{\left(30\sqrt{3}\right)^2 + (-10)^2}.
\]

Calculando:
\[
d(4) = \sqrt{2700 + 100} = \sqrt{2800} \approx 52.92 \, \text{m}.
\]

\subsection*{Resultado Final}
A distância entre os barcos no instante \(t = 4 \, \text{s}\) é:
\[
d(4) \approx 52.92 \, \text{m}.
\]


	% SECÇÃO PROBLEMAS


\end{document}
