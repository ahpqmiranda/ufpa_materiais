\section{Desenvolvimento Prático}

Para garantir um processo documentado e reprodutível, o memorial de cálculo foi estruturado utilizando Python, uma linguagem de programação amplamente adotada para análises científicas e de engenharia devido à sua flexibilidade, bibliotecas especializadas e capacidade de automação de cálculos complexos. Python permitiu que os autores mapeassem detalhadamente o desenvolvimento teórico do trocador de calor, criando rotinas de cálculo que facilitam a verificação dos resultados e asseguram que todas as hipóteses e passos intermediários estejam explicitamente documentados. A escolha por Python também contribuiu para a criação de um fluxo de trabalho modular, onde cada cálculo ou etapa teórica do desenvolvimento do trocador de calor foi representado por funções e scripts específicos.

Dentro do memorial, cada seção aborda um aspecto do desenvolvimento do trocador de calor, desde os cálculos preliminares para dimensionamento até a verificação de parâmetros operacionais, como eficiência térmica e transferência de calor. A implementação dos cálculos em Python proporcionou uma maneira prática de estruturar o memorial, permitindo o ajuste de parâmetros em tempo real e facilitando a análise de sensibilidade e a validação de resultados para diferentes condições de operação.

Para validar a integridade dos cálculos e garantir a reprodutibilidade dos procedimentos, os autores aplicaram uma série de testes e verificações automatizadas que permitiram revisar o memorial em busca de inconsistências ou erros. Esta abordagem sistemática, centrada na programação, fornece um documento detalhado, onde cada etapa é justificável e os resultados podem ser replicados, contribuindo significativamente para a transparência e confiabilidade do projeto.

\subsection{Código Comentado}
No desenvolvimento do experimento, a decisão de se usar python se baseou na familiaridade com seus recursos e ao fato de ser uma ferramenta amplamente reconhecida na área de ciência de dados. Foram empregadas diversas bibliotecas especializadas para realizar os cálculos necessários e assegurar a precisão dos resultados. 

Começamos pela importação dos frameworks para processamento:

\begin{center}
	\begin{lstlisting}[language=Python]
		import numpy as np  # para cálculos matemáticos
		import pandas as pd # para manipulação de dados
		import matplotlib.pyplot as plt # para plotar gráficos
\end{lstlisting}

\text{Fonte: Elaborado pelos autores (2024)}
\end{center}

A biblioteca NumPy foi utilizada para operações vetoriais, enquanto o Pandas facilitou o processamento e manipulação dos dados, principalmente na forma de tabelas (DataFrames). sendo essencial para a leitura, filtragem, limpeza, e agregação de dados. \cite{g_2008_manipulating, harris_2020_array}

Com o Matplotlib, a equipe construiu gráficos que permitiram a visualização clara das informações, tanto em valores do Sistema Internacional (SI) quanto em valores de engenharia.  \cite{hunter_2007_matplotlib}\\

Declaração de variáveis:
\begin{center}
	\begin{lstlisting}[language=python]
		# Dados do problema
		T_agua_entrada = 20 # Temperatura de entrada da água, em °C
		T_agua_saida = 26		# Temperatura de saída da água, em °C
		T_diesel_in = 65		# Temperatura da entrda do diesel, em °C
		T_diesel_out = 38		# Temperatura da saida do diesel, em ºC
		m_dot_agua = 1.5		# Vazão mássica da água, em kg/s
		m_dot_diesel = .7		# Vazão mássica do diesel, em kg/s
		rho_diesel = 0.850		# kg/l
		v_dot_diesel = m_dot_diesel / rho_diesel * 60 # vazão volumétrica do disel
		D_interno = 0.014		# Diâmetro interno do tubo, em metros
		U = 640		# Coeficiente global de transferência de calor, em W/m²·K
		# Calores específicos
		c_p_agua = 4.18		# Calor específico da água, em kJ/kg·K
		c_p_diesel = 2		# Calor específico do óleo diesel, em kJ/kg·K
		# Convertendo calor específico de kJ para J
		c_p_agua *= 1000		# em J/kg·K
		c_p_diesel *= 1000		# em J/kg·K
	\end{lstlisting}
	\text{Fonte: Elaborado pelos autores (2024)}
\end{center}

O trecho a seguir verifica as capacidades térmicas da água e do óleo diesel, para estabelecer a constante \(c\) de dimensionamento de trocas térmicas:

\begin{center}
	\begin{lstlisting}[language=python]
		# Passo 1: Determinar C_h e C_c
		C_h = m_dot_diesel * c_p_diesel # Taxa de capacidade térmica do fluido quente,␣
		↪em W/K
		C_c = m_dot_agua * c_p_agua # Taxa de capacidade térmica do fluido frio, em W/K
	\end{lstlisting}
	\text{Fonte: Elaborado pelos autores (2024)}
\end{center}

A seguir, é realizado o cálculo da constante C, através da verificação das funções máximo e mínimo no python:
\begin{center}
	\begin{lstlisting}[language=python]
		# Identificar o menor valor entre C_h e C_c
		C_min = min(C_h, C_c)
		C_max = max(C_h, C_c)
		c = C_min / C_max
	\end{lstlisting}
	\text{Fonte: Elaborado pelos autores (2024)}
\end{center}

Em seguida, verificamos a taxa máxima de transferência térmica e a taxa real:

\begin{center}
	\begin{lstlisting}[language=python]
		# Passo 3: Calcular a taxa máxima de transferência de calor (Q_max)
		# Temperatura de entrada do fluido quente, em °C
		# Temperatura de entrada do fluido frio, em °C
		Q_max = C_min * (T_diesel_in - T_agua_entrada)
		
		# Passo 4: Calcular a taxa real de transferência de calor (Q)
		Q_real = m_dot_agua * c_p_agua * (T_agua_saida - T_agua_entrada)
	\end{lstlisting}
	\text{Fonte: Elaborado pelos autores (2024)}
\end{center}

A seguir, temos o cálculo da efetividade e do NTU


\begin{center}
	\begin{lstlisting}[language=python]
		# Passo 5: Calcular a efetividade
		efetividade = Q_real / Q_max
		
		# Passo 6: Calcular o NTU usando a fórmula para trocador de calor contracorrente
		NTU = (1 / (c - 1)) * math.log((efetividade - 1) / (efetividade * c - 1))
	\end{lstlisting}
	\text{Fonte: Elaborado pelos autores (2024)}
\end{center}

E por último, calculamos as dimensões de área e comprimento total de tubulação:



\begin{center}
	\begin{lstlisting}[language=python]
		# Passo 7: Calcular a área de transferência de calor (A_s)
		A_s = (NTU * C_min) / U
		# Passo 8: Calcular o comprimento do tubo (L)
		L = A_s / (math.pi * D_interno)
	\end{lstlisting}
	\text{Fonte: Elaborado pelos autores (2024)}
\end{center}

Os resultados obtidos nos cálculos, foram:
\begin{itemize}
	\item $C_{min}: 1400.00 \, \mathrm{W/K}$
	\item $C_{max}: 6270.00 \, \mathrm{W/K}$
	\item $Q_{max}: 63000.00 \, \mathrm{W}$
	\item $Q_{real}: 37620.00 \, \mathrm{W}$
	\item Efetividade: $0.597$
	\item NTU: $0.986$
	\item Área de transferência de calor $(A_s): 2.16 \, \mathrm{m^2}$
	\item Comprimento do tubo $(L): 49.05 \, \mathrm{m}$
\end{itemize}



\begin{center}
	\begin{lstlisting}[language=python]
		conteúdo...
	\end{lstlisting}
	\text{Fonte: Elaborado pelos autores (2024)}
\end{center}
