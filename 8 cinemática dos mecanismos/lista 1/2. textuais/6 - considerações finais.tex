\section{Considerações Finais}

O desenvolvimento do projeto do trocador de calor e os cálculos realizados foram essenciais para compreender as condições operacionais e a demanda térmica exigida pelo sistema. Através desses cálculos, foi possível estabelecer os parâmetros fundamentais para o projeto, como a área de troca térmica, a vazão de óleo e os limites de potência térmica, assegurando que as especificações atendam aos requisitos de eficiência e desempenho esperados.

Entretanto, ao avaliar as opções de trocadores de calor disponíveis, verificou-se que nenhum dos modelos pesquisados atendia integralmente às necessidades do projeto. Este cenário destaca a importância de realizar uma nova pesquisa de mercado para identificar alternativas que possuam especificações mais alinhadas às demandas do sistema. A seleção de um trocador de calor que atenda plenamente aos critérios estabelecidos é essencial para garantir a operação eficiente e sustentável do processo.

Dessa forma, os resultados obtidos até o momento servem como uma base sólida para orientar essa pesquisa de mercado, possibilitando uma comparação mais precisa entre as opções disponíveis e aumentando as chances de encontrar um equipamento que satisfaça todas as exigências do projeto.

\section{Conclusão}

O projeto do trocador de calor, fundamentado em cálculos rigorosos e parâmetros específicos, proporcionou uma compreensão detalhada das necessidades térmicas e operacionais do sistema. Embora os critérios estabelecidos tenham sido cuidadosamente definidos para garantir a eficiência e a confiabilidade do equipamento, a análise revelou que nenhum dos trocadores de calor disponíveis atendia plenamente às demandas do projeto. Portanto, é necessário realizar uma nova pesquisa de mercado para identificar alternativas que melhor correspondam aos requisitos estabelecidos, assegurando que o trocador selecionado seja capaz de operar de forma eficaz nas condições especificadas.
