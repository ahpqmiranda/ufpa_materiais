\section{Metodologia}
Para os trocadores de calor, diversos métodos têm sido publicados por diferentes autores ao longo dos anos. Os mais tradicionais e amplamente utilizados são os da média logarítmica da diferença de temperaturas (MLDT) e efetividade – NUT (εNUT). Basicamente, estes métodos exploram a razão entre a taxa de transferência de calor real e a taxa máxima possível. Posteriormente, outras linhas de pesquisa foram abordadas, como a análise de geração de entropia por SeКulic (1990), o conceito de eficiência por \cite{orgeda_2020_trocadores} e o método do princípio da uniformidade do campo da diferença de temperaturas proposto \cite{cengel_2012_transferencia}

O método da efetividade – NUT (ε-NUT) é amplamente utilizado em situações onde o tamanho do trocador de calor e as temperaturas de entrada são conhecidos e a taxa de transferência de calor e as temperaturas de saída dos fluidos são pretendidas. Problemas de dimensionamento também podem ser solucionados através deste método. Em resumo, o método ε-NUT pode ser definido como a razão entre a taxa de transferência de calor real do trocador de calor em estudo e a taxa de transferência máxima, que pode ser obtida de um trocador de calor contracorrente puro com comprimento infinito, o que garante a máxima diferença possível de temperaturas no fluido de menor capacidade térmica. Esta razão pode então ser escrita da seguinte forma: \cite{stenstrasser_2017_projeto, abdallah_2018_multi}

\begin{equation}
	\varepsilon = \frac{q}{q_{\text{máx}}} = \frac{C_q (T_{q,\text{ent}} - T_{q,\text{sai}})}{C_{\text{min}}(T_{q,\text{ent}} - T_{f,\text{ent}})} = \frac{C_f (T_{f,\text{sai}} - T_{f,\text{ent}})}{C_{\text{min}}(T_{q,\text{ent}} - T_{f,\text{ent}})}
\end{equation}
De acordo com \citen{abdallah_2018_multi}, a efetividade de todo trocador de calor pode ser expressa em termos do número de unidades de transferência (NUT) e da razão entre as capacidades térmicas dos fluidos C*. 

A partir de então, fórmulas específicas foram desenvolvidas para os principais tipos e arranjos de trocadores de calor, como pode ser consultado em \citenum{stenstrasser_2017_projeto} para casos com escoamentos em paralelo, contracorrente, tipo casco-tubo e com escoamentos cruzados, objeto de estudo do presente em \citen{stenstrasser_2017_projeto} encontram-se dois métodos para a análise dos trocadores de calor: média logarítmica das diferenças de temperatura e o método da efetividade. 

Para calcular o desempenho de um trocador é necessário relacionar a taxa total de transferência de calor, as temperaturas de entrada e saída dos fluidos, o coeficiente global de transferência de calor e a área total disponível para a troca térmica. A taxa de transferência de calor pode ser determinada pelas equações abaixo:
\\
\begin{equation}
	q = m_q \cdot C_{p,q} (T_{q,e} - T_{q,s})
\end{equation}

\begin{equation}
	q = m_f \cdot C_{p,f} (T_{f,e} - T_{f,s})
\end{equation}

Onde os índices \(q \) e \(f\), referem-se aos fluidos quentes e frios, \(e\) e \(s\) representam a entrada e a saída respectivamente. As vazões das correntes são definidas como \(m\), \(C_p\) é a capacidade calorífica e \(T\) as temperaturas.
A diferença de temperatura (\(\Delta T\)) entre os fluidos quente e frio é dada por:

\begin{equation}
	\Delta T \equiv T_q - T_f
\end{equation}

\begin{equation}
	\Delta T_1 = T_{q,e} - T_{f,s}
\end{equation}

\begin{equation}
	\Delta T_2 = T_{q,s} - T_{f,e}
\end{equation}

Quando as temperaturas não são conhecidas ou especificadas, é preferível utilizar o método da efetividade-NUT ($\varepsilon -$  NUT). Nesse método, calcula-se primeiramente o máximo calor trocado. Para isso tem-se:


\begin{equation}
	C_f < C_q \quad q_{\text{máx}} = C_f (T_{q,i} - T_{f,i})
\end{equation}

\begin{equation}
	C_q < C_f \quad q_{\text{máx}} = C_q (T_{q,i} - T_{f,i})
\end{equation}


Onde \(C_f\) e \(C_q\) e são as vazões mássicas multiplicadas pelo calor específico correspondente dos fluidos frio e quente, respectivamente.
Com base nas equações anteriores, podemos escrever a seguinte equação:

\begin{equation}
	q_{max} = C_{min}  (T_{h,i} - T_{c,i})
\end{equation}

O conhecimento da efetividade é útil, pois a taxa real de transferência de calor pode ser determinada de imediato:

\begin{equation}
	q = \varepsilon C_{min} (T_{q,e} - T_{f,e})
\end{equation}

O número de unidades de transferência (NUT) é um parâmetro adimensional utilizado para a análise de trocadores de calor, definido como:

\begin{equation}
	\text{NUT} = \frac{UA}{C_{min}}
\end{equation}

Foram desenvolvidas equações que determinam de forma específica a relação efetividade-NUT. Para trocadores de calor duplo tubo em paralelo tem-se:\\

\textbf{A) Parallel flow}
\[
\varepsilon = \frac{1 - \exp \left[ -\text{NTU} (1 + C_r) \right]}{1 + C_r}
\]

\textbf{B) Counterflow}
\[
\varepsilon = \frac{1 - \exp \left[ -\text{NTU} (1 - C_r) \right]}{1 - C_r \exp \left[ -\text{NTU} (1 - C_r) \right]} \quad (C_r < 1)
\]

Onde, 

\begin{equation}
	C_r = \frac{C_{min}}{C_{max}}
\end{equation}

No caso de contracorrente:

\[
\varepsilon = \frac{1 - \exp \left[-\text{NUT} (1 - C_r)\right]}{1 - C_r \exp \left[-\text{NUT} (1 - C_r)\right]} \quad (C_r < 1)
\]

\[
\varepsilon \equiv \frac{\text{NUT}}{1 + \text{NUT}} \quad (C_r = 1)
\]


Em cálculos envolvendo o projeto de trocadores de calor contracorrente, é mais conveniente trabalhar com relação ε-NUT na forma:
\begin{equation}
	\text{NUT} = \frac{1}{C_r - 1} \ln \left( \frac{\varepsilon - 1}{\varepsilon \cdot C_r - 1} \right) \quad (C_r < 1)
\end{equation}

\begin{equation}
	\text{NUT} \equiv \frac{\varepsilon}{1 - \varepsilon} \quad (C_r = 1)
\end{equation}
