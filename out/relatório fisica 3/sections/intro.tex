\section{Introdução}
\label{Introducao}

    \indent O circuito proposto é mostrado na Figura 1. Como parâmetros iniciais, determinou-se os valores abaixo.

\begin{figure}[h!]
    \centering
    \begin{circuitikz}
        \draw (0,0)
        (0,0) node[npn, tr circle](Q1){BC548}
        (Q1.base) node[anchor=east]{B}
        (Q1.collector) node[anchor=south]{C}
        (Q1.emitter) node[anchor=north]{E}
        (-2,0) -- (-1,0)
        (-2,0) to [R, l=10K{$\Omega$}] (-2,3.5)
        (0,0.9) to [R, l=3.9K{$\Omega$}] (0,3.5)
        (-2,3.5) -- (0,3.5)
        (-1, 3.5) -- (-1, 4) node[vcc]{$V_{cc}$}
        (-2,0) to [R, l=2.2K{$\Omega$}] (-2,-3.5)
        (-2,-3.5) -- (2, -3.5)
        (0, -0.9) to [eC, l=50<\mu\farad>] (0, -3.5)
        (0, -1) -- (2, -1)
        (2,-1) to[R, l=\SI{820}{$\Omega$}] (2,-3.5)
        (0,-3.5) -- (0,-4.5) node[tlground]{GND}
        (-2,0) to [eC, *-o, l=10<\mu\farad>] (-4,0) node[left]{IN}
        (0,0.9) to [eC, *-o, l=10<\mu\farad>] (2,0.9)  node[right]{OUT}
        ;
    \end{circuitikz}\label{fig:figure}
    \caption{Circuito proposto}

\end{figure}




