\section[Planejamento]{Objetivos}\label{sec:objetivos}
\indent O objetivo deste relatório é descrever os procedimentos realizados em laboratório para o experimento 8, cujo o objetivo foi iniciar análises relacionados a termoquímica, utilizando
a lei de Hess, comparando o comportamento teórico com a prática, e gerando uma discussão sobre os resultados obtidos.

\indent A primeira lei da termodinâmica é a lei da conservação da energia, que afirma que a energia total de um sistema é constante. A segunda lei da termodinâmica é a lei da entropia, que afirma que o aumento da entropia de um sistema sempre aumentará até um valor máximo. A terceira lei, também conhecida como a lei da zero da termodinâmica, entropia, que afirma que a entropia de um sistema não pode ser negativa.

\indent A quantidade de calor envolvida em uma reação química é medida pelo calor de reação, ou $q$. O calor de reação é a quantidade de calor liberado ou absorvido pelo sistema quando uma reação ocorre. O calor de reação pode ser calculado usando a equação $q = ΔH$, onde $ΔH$ é a mudança de entalpia da reação. A mudança de entalpia é a diferença entre a energia final do sistema e a energia inicial do sistema.

\indent No exemplo abaixo, a reação é a decomposição do ácido sulfúrico em água e gás sulfúrico.


\begin{center}
    \schemestart $2 H_2SO_4(aq)$ \arrow{->} $2 H_2O(l) + SO_2(g)$ \schemestop
\end{center}

Neste exemplo, $q = ΔH = -184 kJ$. Isto significa que 184 kJ de energia serão liberados quando 2 mol de ácido sulfúrico se decompuser em água e gás sulfúrico.