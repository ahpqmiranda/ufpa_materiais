    \section{Introdução}\label{sec:intro}
    
    \indent A manipulação de ácidos e bases são parte do cotidiano de um químico, e é importante que o mesmo saiba como identificar a presença de um ácido ou base em uma solução, para que possa tomar as medidas de segurança necessárias.\  A utilização de indicadores ácidos e bases é uma forma de identificar a presença de um ácido ou base em uma solução, sem a necessidade de realizar uma titulação.\  Os indicadores ácidos e bases são substâncias que mudam de cor quando expostas a uma solução ácida ou básica, respectivamente.\  A cor de um indicador ácido ou base é chamada de cor de transição, e é a cor que o indicador apresenta quando exposto a uma solução ácida ou básica.\  A cor de transição de um indicador ácido ou base é determinada por sua estrutura química, e é uma característica que não pode ser alterada pelo pH da solução.
    \indent O objetivo deste relatório é identificar a presença de ácidos e bases em soluções, utilizando indicadores ácidos e bases, e determinar a cor de transição de cada indicador utilizado, seguindo as orientações do material de apoio do professor.




