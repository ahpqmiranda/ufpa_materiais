%! Author = alana
%! Date = 18/09/2022

% Preamble
\documentclass[a4paper, 11pt]{article}

% Packages
\usepackage[brazil]{babel}
\usepackage[utf8]{inputenc}
\usepackage{amsmath, lmodern, amsthm, amstext, ebezier, amscd}
\usepackage{hyperref}
\usepackage{graphicx}
\usepackage[a4paper, left=2.5cm, right=2.5cm, top=2.5cm, bottom=2.5cm]{geometry}
\usepackage{setspace} \onehalfspacing
\usepackage{indentfirst}
\usepackage{listings}


% Document
\begin{document}
\thispagestyle{empty}

    \begin{center}
        \parbox{3cm}{\includegraphics[scale=1]{logo_ufpa}}\\
        \vspace{1cm}
        \Large \uppercase{Universidade Federal do Pará}\\
        \Large \uppercase{Instituto de Tecnologia - ITEC}\\
        \vspace{2.5cm}
        \Large \uppercase{Faculdade de Engenharia Mecânica - FEM}
        \Large \uppercase{Laboratório de Química Analítica Quantitativa - LQAQ}\\
        \vspace{2.5cm}
        \Large \textbf{\uppercase {Relatório de Prática 2: Reações de Dupla Troca}} \\
        \Large \textbf{\uppercase {PROF. DR. Carlos Antônio Neves}} \\
        \vspace{2.5cm}
        \Large \uppercase {Alan Henrique Pereira Miranda - 202102140072}\\
        \Large \uppercase {Gabriel Cruz de Oliveira - 202102140055}\\
        \Large \uppercase {Paloma Gama da Silva - 202102140029}\\
        \Large \uppercase {Silvio Farias Leal - 202102140035}\\
        \vspace{1cm}
        \Large {Belém-PA \\ 2022}

    \end{center}

\newpage
\section{Introdução}\label{sec:intro}
\indent O objetivo deste relatório é apresentar os resultados obtidos na prática de reações de dupla troca, realizada no dia 18/09/2022, no laboratório de Química Analítica Quantitativa da Universidade Federal do Pará.
    A prática consistiu em realizar a troca de íons entre dois sais, sendo um deles um sal de cobre e o outro um sal de prata. A reação de dupla troca é uma reação química em que dois compostos químicos trocam íons entre si, formando dois novos compostos. A reação de dupla troca é uma reação de troca de íons, ou seja, a troca de íons entre dois compostos químicos.




\end{document}