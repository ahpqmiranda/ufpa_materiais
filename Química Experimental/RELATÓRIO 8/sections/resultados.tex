\section{Discussão e Resultados}\label{sec:discussao}

    \indent O procedimento de cálculo teórico dos calores associados a formação dos compostos utilizados foi de relativa complexidade, dada a ausência de fontes primárias e confiáveis, porém, foi adotado como base a fonte secundária \cite{TABELA_ENTALPIAS}, que apresenta uma boa precisão.\\

    \indent A realização das experiências foi de certa complexidade a equipe, devido a imperícia na leitura da temperatura no termômetro e no momento adequado de remoção do termômetro da solução, o que pode ter causado uma variação de temperatura não esperada, e consequentemente, uma variação no calor experimental das reações.\\

    \indent A disparidade entre os valores obtidos teoricamente e experimentalmente pode ser explicada pela imperícia na execução, falta de tato com relação aos tempos de medição e a imprecisão dos ferramentais utilizados, e consequentemente, uma variação no calor experimental das reações.\\

    \indent Os erros aproximados foram calculados utilizando a fórmula $E = \frac{V_{exp} - V_{teo}}{V_{teo}}$, onde $V_{exp}$ é o valor experimental e $V_{teo}$ é o valor teórico.\ Os erros foram calculados para os valores de calor de reação e calor de solubilização.\ Os erros foram de $-67,69\%$ e $-56,19\%$ para os valores de calor de reação e calor de solubilização, respectivamente.\

    \begin{equation}
        \begin{split}
            \notag
            $Erro_{exp1} = \frac{\Delta H_{exp1} - \Delta H_{teo1}}{\Delta H_{teo1}}$
            $ &= \frac{0,882 - 2,73}{2,73} \cdot 100\% $
            $ &=  -67,69\% $\\
        \end{split}
    \end{equation}

    \begin{equation}
        \begin{split}
            \notag
            $Erro_{exp2} = \frac{\Delta H_{exp2} - \Delta H_{teo2}}{\Delta H_{teo2}}$
            $ &= \frac{3,12 - 2,46}{2,46} \cdot 100\% $
            $ &=  26,83\% $ \\
        \end{split}
    \end{equation}\\

    \indent É possível observar que os erros foram de grande magnitude, o que faria ser necessário a revisão de conceitos e de procedimentos realizados como forma de minimizar a disparidade entre os valores obtidos.\\

