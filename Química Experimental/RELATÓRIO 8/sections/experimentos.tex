\section[Parte Experimental]{Determinação do Calor de Reação na Dissolução de NaOH em HCl}\label{sec:parte_experimental}
    \subsection{Procedimento}\label{sec:procedimento}

        \indent Para a realização desta prática, foram utilizados os materiais e utensílios descritos na seção \ref{sec:mat_materiais}.\ O procedimento para a medição do calor de dissolução do NaOH foi indireto, ou seja, não foi medido diretamente, para tal, foram preparadas duas soluções, uma de NaOH 0,5 mol/L, uma com 50 mL e outra com 50 mL de HCl a 0,5 mol/L.\ Tal reação possui balanço estequiométrico 1:1:1:1, portanto o procedimento de medição das entalpias associadas é mais simplificado.\\
        
        \indent A primeira solução foi preparada em um erlenmeyer de 250 mL, que foi colocado dentro do copo de isopor para reduzir a perda térmica da reação para o ambiente.\ Foi adicionado 50 mL de NaOH 0,5 mol/L e 50 mL de HCl 0,5 mol/L, e por fim, agitou-se a solução até que a reação fosse completa, conforme a equação abaixo:
        \begin{equation}
            \ce{NaOH(aq) + HCl(aq) -> NaCl(aq) + H2O(l)}
        \end{equation}
    
    	\indent As condições observadas durante o procedimento foram:
    	\begin{itemize}
    		\item Temperatura ambiente: $20ºC$ (a confirmar)
    		\item Temperatura da solução: $25.1ºC$ (a confirmar)
            \item Volume da solução: $100 mL$
            \item A variação de temperatura foi de $5.1ºC$
    	\end{itemize}

        \indent Por seguinte, a amostra de NaOH sólido foi pesada e colocada em um erlenmeyer de 250 mL. Em seguida, foi adicionada 50 mL de solução de HCl 0,50 mol/L. O calor de reação foi medido utilizando um termômetro de vidro.
        \begin{equation}
            \ce{NaOH(s) + HCl(aq) -> NaCl(aq) + H2O(l)}
        \end{equation}

        \indent As condições observadas durante o procedimento foram:
        \begin{itemize}
            \item Temperatura ambiente: $20ºC$ (a confirmar)
            \item Temperatura da solução: $29.5ºC$ (a confirmar)
            \item Volume da solução: $50 mL$
            \item A variação de temperatura foi de $9.5ºC$ (a confirmar)
        \end{itemize}
         A temperatura inicial foi de $25,0°C$ e a temperatura final foi de $28,0°C$. A variação de temperatura foi de $3,0°C$.\\
