\section[Parte Experimental]{Determinação dos Calores de Reação  $\Delta{H_{1}}$ e Dissolução $\Delta{H_{2}}$ na neutralização de NaOH com HCl}\label{sec:parte_experimental}
    \subsection{Procedimento}\label{sec:procedimento}

        \indent Para a realização desta prática, foram utilizados os materiais e utensílios descritos na seção \ref{sec:mat_materiais}.\ O procedimento para a medição do calor de dissolução do NaOH foi indireto, ou seja, não foi medido diretamente, para tal, foram preparadas duas soluções, a primeira, foi uma de NaOH 0,5 mol/L de 50 mL e, que posteriormente foi misturada com outra solução com 50 mL de HCl a 0,5 mol/L.\ Enquanto que a segunda solução foi preparada com 50 mL de HCl a 0,5 mol/L, que posteriomente recebeu 20g de NaOH sólido.\ Tal reação possui balanço estequiométrico 1:1:1:1, portanto o procedimento de medição das entalpias associadas é mais simplificado.\\
        
        \indent A primeira solução foi preparada em um Erlenmeyer de 250 mL, que foi colocado dentro do copo de isopor para reduzir a perda térmica da reação para o ambiente.\ Foram fracionadas porções das soluções em duas provetas, com 50 mL de NaOH 0,5 mol/L e 50 mL de HCl 0,5 mol/L, respectivamente. E por fim, ambas as soluções foram adicionadas ao Erlenmeyer, onde também foi colocado um termômetro de mercúrio para verificar a elevação da temperatura pela reação.\ Agitou-se a solução até que a reação fosse completa, conforme a equação abaixo:
        \begin{equation}
            {NaOH(aq) + HCl(aq) -> NaCl(aq) + H2O(l)}
        \end{equation}
    
    	\indent As condições observadas durante o procedimento foram:
    	\begin{itemize}
    		\item Temperatura ambiente: $20^{\circ}C$ \textit{(a confirmar)}
    		\item Temperatura da solução: $25.1^{\circ}C$ \textit{(a confirmar)}
            \item Volume da solução final: $100 mL$
            \item A variação de temperatura foi de ${5.1}^{\circ}C$ \textit{(a confirmar)}
    	\end{itemize}

        \indent Por seguinte, a amostra de NaOH sólido foi pesada em $20g$ e colocada em um erlenmeyer de 250 mL isolado pelo copo de isopor. Em seguida, foi adicionada 50 mL de solução de HCl 0,50 mol/L. O calor de reação foi medido utilizando um termômetro de vidro, e a solução foi agitada até que o soluto adicionado fosse completamente dissolvido.\ A reação realizada foi a seguinte:\
        \begin{equation}
            {NaOH(s) + HCl(aq) -> NaCl(aq) + H2O(l)}
        \end{equation}
    	
    	\indent As condições observadas durante o procedimento foram:
    	\begin{itemize}
    		\item Temperatura ambiente: $20^{\circ}C$ \textit{(a confirmar)}
    		\item Temperatura da solução: $29.5^{\circ}C$ (\textit{(a confirmar)}
    		\item Volume da solução: $50 mL$
    		\item A variação de temperatura foi de $9.5^{\circ}C$ \textit{(a confirmar)}
    	\end{itemize}
    
    	\indent A temperatura inicial foi de $20,0^{\circ} C$ e a temperatura final foi de $29,3^{\circ}C$. A variação de temperatura foi de $9,3^{\circ} C$.\\
                

        \subsection{Cálculos}\label{sec:calculos}
            \indent A equação da entalpia de dissolução de NaOH em HCl é dada por:
            \begin{equation}
                \Delta H_{rxn} = \Delta H_{for} + \Delta H_{dis} - \Delta H_{Neutr}
			\end{equation}
			\indent A legenda para equação é: $\Delta H_{rxn}$: entalpia da reação; $\Delta H_{fus}$: entalpia de formação; $\Delta H_{dis}$: entalpia de dissolução; $\Delta H_{Neutr}$: entalpia de neutralização.\\
			\indent E onde $\Delta H_{rxn}$: entalpia da reação, sendo descrita como:\
			\begin{equation}
				\Delta H_{dis} = H_{inicial} - H_{final}
			\end{equation}
			\indent \textbf{OBSERVAÇÂO}: Não foram encontradas tabelas de entalpia de formação para o Hidróxido de Sódio e para o Ácido Cloridrico a $20^{\circ}C$, mas sim a
            $25^{\circ}C$, portanto, foi feito um ajuste de temperatura para $20^{\circ}C$ utilizando a equação de Arrhenius, conforme a equação abaixo:
            \begin{equation}
                \Delta H_{rxn} = \Delta H_{rxn}^{25^{\circ}C} - \frac{E_{a}}{R} \left( \frac{1}{T} - \frac{1}{298} \right)
            \end{equation}
            \indent Onde $\Delta H_{rxn}^{25^{\circ}C}$: entalpia de reação a $25^{\circ}C$; $E_{a}$: energia de ativação; $R$: constante dos gases ideais; $T$: temperatura; $298$: temperatura de referência.\\
            \indent A entalpia de formação do Hidróxido de Sódio foi encontrada na tabela de entalpia de formação de substâncias inorgânicas, conforme a tabela abaixo:
            \begin{table}[h]
                \centering
                \begin{tabular}{|c|c|c|}
                    \hline
                    \textbf{Substância} & \textbf{Entalpia de Formação} & \textbf{Unidade} \\
                    \hline
                    Hidróxido de Sódio & $- 1.1 \times 10^{3}$ & $kJ \cdot mol^{-1}$ \\
                    \hline
                    Ácido Clorídrico & $- 1.1 \times 10^{3}$ & $kJ \cdot mol^{-1}$ \\
                    \hline
                \end{tabular}
                \caption{Entalpia de Formação do Hidróxido de Sódio}
                \label{tab:enthalpy}
            \end{table}
            \indent O calor de fusão do NaOH é de $6,02 kJ/mol$ e o calor de vaporização do H2O é de $40,66 kJ/mol$.\\
            \indent O calor de dissolução do NaOH em HCl é de $- 46,68 kJ/mol$.\\
            \indent O calor de dissolução do NaOH sólido em HCl é de $- 52,70 kJ/mol$.\\
            \indent A variação de entalpia de dissolução do NaOH sólido em HCl é de $- 6,02 kJ/mol$.\\
            
            \subsubsection{Determinação do Calor de Neutralização $\Delta{H_{neu}}$ do NaOH em HCl}\label{sec:determinacao_calor_neutraliza}
            \indent A equação da entalpia de dissolução de NaOH em HCl é dada por:\\
            
            \subsubsection{Determinação do Calor de Dissolução $\Delta{H_{dis}}$ do NaOH em HCl}\label{sec:determinacao_calor_dissolucao}
            \indent A equação da entalpia de dissolução de NaOH em HCl é dada por:\\


