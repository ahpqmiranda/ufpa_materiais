\section[Parte Experimental]{Determinação dos Calores de Reação  $\Delta{H_{1}}$ e Dissolução $\Delta{H_{2}}$ na neutralização de NaOH com HCl}\label{sec:parte_experimental}
    \subsection{Procedimento}\label{sec:procedimento}

        \indent Para a realização desta prática, foram utilizados os materiais e utensílios descritos na seção \ref{sec:mat_materiais}.\ O procedimento para a medição do calor de dissolução do NaOH foi indireto, ou seja, não foi medido diretamente, para tal, foram preparadas duas soluções, a primeira, foi uma de NaOH 0,5 mol/L de 50 mL e, que posteriormente foi misturada com outra solução com 50 mL de HCl a 0,5 mol/L.\ Enquanto que a segunda solução foi preparada com 50 mL de HCl a 0,5 mol/L, que posteriomente recebeu 20g de NaOH sólido.\ Tal reação possui balanço estequiométrico 1:1:1:1, portanto o procedimento de medição das entalpias associadas é mais simplificado.\\
        
        \indent A primeira solução foi preparada em um Erlenmeyer de 250 mL, que foi colocado dentro do copo de isopor para reduzir a perda térmica da reação para o ambiente.\ Foram fracionadas porções das soluções em duas provetas, com 50 mL de NaOH 0,5 mol/L e 50 mL de HCl 0,5 mol/L, respectivamente. E por fim, ambas as soluções foram adicionadas ao Erlenmeyer, onde também foi colocado um termômetro de mercúrio para verificar a elevação da temperatura pela reação.\ Agitou-se a solução até que a reação fosse completa, conforme a equação abaixo:
        \begin{equation}
            {NaOH(aq) + HCl(aq) -> NaCl(aq) + H2O(l)}
            \label{eq:equação_principal_aquosa}
        \end{equation}
    
    	\indent As condições observadas durante o procedimento foram:
    	\begin{itemize}
    		\item Temperatura ambiente: $20.5^{\circ}C$ 
    		\item Temperatura das soluções pré experimento:  $23.4^{\circ}C$
    		\item Temperatura da solução: $25.1^{\circ}C$ \textit{(a confirmar)}
            \item Volume da solução final: $100 mL$
            \item A variação de temperatura foi de ${5.1}^{\circ}C$ \textit{(a confirmar)}
    	\end{itemize}

        \indent Por seguinte, a amostra de NaOH sólido foi pesada em $20g$ e colocada em um erlenmeyer de 250 mL isolado pelo copo de isopor. Em seguida, foi adicionada 50 mL de solução de HCl 0,50 mol/L. O calor de reação foi medido utilizando um termômetro de vidro, e a solução foi agitada até que o soluto adicionado fosse completamente dissolvido.\ A reação realizada foi a seguinte:\
        \begin{equation}
            {NaOH(s) + HCl(aq) -> NaCl(aq) + H2O(l)}
            \label{eq:equação principal NaOH sólido}
        \end{equation}
    	
    	\indent As condições observadas durante o procedimento foram:
    	\begin{itemize}
    		\item Temperatura ambiente: $20^{\circ}C$ \textit{(a confirmar)}
    		\item Temperatura da solução: $29.5^{\circ}C$ (\textit{(a confirmar)}
    		\item Volume da solução: $50 mL$
    		\item A variação de temperatura foi de $9.5^{\circ}C$ \textit{(a confirmar)}
    	\end{itemize}
    
    	\indent A temperatura inicial foi de $20,0^{\circ} C$ e a temperatura final foi de $29,3^{\circ}C$.\ A variação de temperatura foi de $9,3^{\circ} C$.\\
                

        \subsection{Cálculos}\label{sec:calculos}
            \indent A equação da entalpia de dissolução de NaOH em HCl é dada por:
            \begin{equation}
                \Delta H_{rxn} = \Delta H_{for} + \Delta H_{dis} - \Delta H_{Neu}\label{eq:equacao de entalpia geral}
            \end{equation}
			\indent A legenda para equação é: $\Delta H_{rxn}$: entalpia da reação; $\Delta H_{fus}$: entalpia de formação; $\Delta H_{dis}$: entalpia de dissolução; $\Delta H_{Neu}$: entalpia de neutralização.\\
			\indent E onde $\Delta H_{rxn}$: entalpia da reação, sendo descrita como:\
			\begin{equation}
				\Delta H_{dis} = H_{inicial} - H_{final}\label{eq:equation1}
            \end{equation}
			\indent \textbf{OBSERVAÇÂO}: Não foram encontradas tabelas de entalpia de formação para o Hidróxido de Sódio e para o Ácido Cloridrico a $20^{\circ}C$, mas sim a
            $25^{\circ}C$, portanto, foram considerados ajustes nos valores de entalpia a partir da temperatura para $20^{\circ}C$ utilizando a equação de Arrhenius, porém, após realizar os cálculos, as correções nos valores foram tão infimos que não foram considerados no resultado final.\\
            
            \indent A entalpia de formação do Hidróxido de Sódio foi encontrada na tabela de entalpia de formação de substâncias inorgânicas, conforme a tabela abaixo:



             \begin{table}[h]
                 \centering
                 \begin{tabular}{>{\RaggedRight} {3.50cm} p{5.50cm} p{5.50cm}}
                     \toprule
                     \multicolumn{1}{1}{\textbf{Substância}} & \textbf{Entalpia de Formação} & \textbf{Unidade}\\
                     \midrule
                     Hidróxido de Sódio sólido & $4,27 \times 10^{2}$ & $kJ \cdot mol^{-1}$ \\
                    Hidróxido de Sódio aquoso & $4,70 \times 10^{2}$ & $kJ \cdot mol^{-1}$ \\
                    Ácido Clorídrico aquoso & $1,67 \times 10^{2}$ & $kJ \cdot mol^{-1}$ \\
                    Ânion Cloreto aquoso & $2,42 \times 10^{2}$ & $kJ \cdot mol^{-1}$ \\
                    Cátion Hidrogênio aquoso & $2,36 \times 10^{2}$ & $kJ \cdot mol^{-1}$ \\
                     \bottomrule
                 \end{tabular}
                 \caption{Entalpia de Formação das substâncias utilizadas.}
                 \label{tab:Tabela de entalpias de formação}
             \end{table}


            \subsubsection{Determinação do Calor de Neutralização $\Delta{H_{neu}}$ do NaOH em HCl}\label{subsubsec:determinacao_calor_neutraliza}
            \indent O primeiro passo do procedimento é quantificar o calor liberado tanto com os resultados obtidos pela literatura, quanto os resultados obtidos experimentalmente.\
            
            \indent A equação a ser trabalhada, é a equação \ref{eq:equação_principal_aquosa}, onde buscaremos as entalpias de formação de cada composto para descobrirmos o calor de neutralização, portanto, temos:\
        
        	\indent As entalpias de formação associadas a cada composto, são:
        	\begin{enumerate}
                \notag
        		\item  $Na^{+}_{(aq)} + OH^{-}_{(aq)} \rightarrow NaOH_{(aq)} \qquad \Delta H_{NaOH} = 470,7 kJ \cdot mol^{-1}$
                \item $H^{+}_{(aq)} + Cl^{-}_{(aq)} \rightarrow HCl_{(aq)} \qquad \Delta H_{HCl} = 167,5 kJ \cdot mol^{-1}$
                \item $Na^{+}_{(aq)} + Cl^{-}_{(aq)} \rightarrow NaCl_{(aq)} \qquad \Delta H_{NaCl} = 407,0 kJ \cdot mol^{-1}$
                \item $H^{+}_{(aq)} + 1/2 \dot O_{2}_{(g)} \rightarrow H_{2}O_{(l)} \qquad \Delta H_{H2O} = 285,8 kJ \cdot mol^{-1}$
        	\end{enumerate}

            \indent Podemos reestruturar as equações a cima para que possamos trabalhar com a equação global \ref{eq:equação_principal_aquosa}, portanto, temos:\
            \begin{table}[h]
            	\centering
            	\renewcommand{\arraystretch}{2}
                \begin{tabular}{ll}
        			$NaOH_{(aq)} \rightarrow Na^{+}_{(aq)} + OH^{-}_{(aq)}$ & $\Delta H_{NaOH} = -470,7 kJ \cdot mol^{-1}$ \\
                    $HCl_{(aq)} \rightarrow H^{+}_{(aq)} + Cl^{-}_{(aq)}$ & $\Delta H_{HCl} = -167,5 kJ \cdot mol^{-1}$ \\
                    $Na^{+}_{(aq)} + Cl^{-}_{(aq)} \rightarrow NaCl_{(aq)}$ & $\Delta H_{NaCl} = +407,0 kJ \cdot mol^{-1}$ \\
                    ${}H^{+}_{(aq)} + (1/2) O_{2}_{(g)} \rightarrow H_{2}O_{(l)}$  & $\Delta H_{H2O} = +285,8 kJ \cdot mol^{-1}$\\
                    \hline
                    $NaOH_{(aq)} + HCl_{(aq)} \rightarrow NaCl_{(aq)} + H_{2}O_{(l)}$ & $\Delta H_{neu} = +54,60 \cdot kJ \cdot mol^{-1}$ \\
				\end{tabular}\label{tab:table}
            \end{table}\\

            \indent O calor de neutralização, $\Delta{H_{neu}}$, é positivo, pois a reação é exotérmica.\\
            \indent A proprção molar em que o experimento foi trabalhado nos reagentes foi de $C = 0.5 \cdot mol \cdot L^{-1}$, a solução possuía um volume $V = 0.1\cdot L$, resultando em um número de mols de $n = 0.05 \cdot mol$, portanto, temos:\\
            \begin{equation}
                \begin{split}
                	\notag
                    \Delta{H_{real_1}} &= \Delta{H_{neu}} \cdot C \cdot V_{solucao}\\
                    \Delta{H_{real_1}} &=(54,6 kJ \cdot mol^{-1}) \cdot (0,50 \cdot mol \cdot L^{-1}) \cedot (0.1 \cedot L) \\
                    &= 2,73 \cdot kJ\\
                \end{split}\label{eq:equation}
            \end{equation}     
            
            \indent Em relação a parte experimental, os valores utilizados foram os seguintes:\\
            \begin{enumerate}
            	\item $m_{soluto} = 3.822 \cdot g$
                \item $m_{solvente} = 100 \cdot g$
                \item $m_{solucao} = 103.822 \cdot g$
                \item $m_{vidro} = 128.192 \cdot g$
                \item $V_{sol} = 0.1 \cdot L$
                \item $T_{inicial} = 23.4 \cdot ^{\circ}C$
                \item $T_{final} = 25.1 \cdot ^{\circ}C$
                \item $c_{solucao} = 4.17 J /g ^{\circ}C$
                \item $c_{vidro} = 0.67 J / g ^{\circ}C$
            \end{enumerate}

            \indent A partir dos dados acima, podemos calcular o calor de neutralização, $\Delta{H_{neu}}$, da seguinte forma:\
            \begin{equation}
                \begin{split}
                	\notag
                    \Delta{H_{neu}} &= \Delta Q_{vidro} + \Delta Q_{solucao} \\
                    \Delta{H_{neu}} &= (m_{vidro} \cdot c_{vidro} \cdot \Delta T) + (m_{solucao} \cdot c_{solucao} \cdot \Delta T) \\
                    \Delta{H_{neu}} &= (m_{vidro} \cdot c_{vidro} + m_{solucao} \cdot c_{solucao}) \cdot \Delta T \\
                    \Delta{H_{neu}} &= (128,192 \cdot 0,67 + 103,822 \cdot 4,17) \cdot 1,7 \cdot J \\
                    \Delta{H_{neu}} &= 0,882 \cdot kJ
                \end{split}
            \end{equation}\\

            \indent Assim, foi possível verificar a entalpia de neutralização de maneira experimental e teórica.\\
        
            
            \subsubsection{Determinação do Calor de Dissolução $\Delta{H_{dis}}$ do NaOH em HCl}\label{subsubsec:determinacao_calor_dissolucao}
            \indent O calor de dissolução do hidróxido de sódio entra como um "extra" do processo que foi analisado dentro da secção \ref{subsubsec:determinacao_calor_neutraliza}.\ Portanto, poucos parâmetros mudam, realizamos a substituição da entalpia de formação do hidróxido de sódio aquoso pela entalpia da mesma substância no estado sólido: $H_{NaOH} = 427 kJ \cdot mol^{-1}$, e sabendo o valor da entalpia de neutralização, passamos a adicionar uma nova incógnita, a $H_{dis}$, esta será negativa pois o processo de dissolução é espontâneo e portanto, exotérmico.\
            
            \indent Portanto, vamos utilizar a lei de Hess para encontrar a entalpia da nova reação:\

            \begin{table}[h]
            	\centering
            	\renewcommand{\arraystretch}{2}
                \begin{tabular}{ll}
        			$NaOH_{(s)} \rightarrow Na^{+}_{(aq)} + OH^{-}_{(aq)}$ & $\Delta H_{NaOH} = -427,0 kJ \cdot mol^{-1}$ \\
                    $HCl_{(aq)} \rightarrow H^{+}_{(aq)} + Cl^{-}_{(aq)}$ & $\Delta H_{HCl} = -167,5 kJ \cdot mol^{-1}$ \\
                    $Na^{+}_{(aq)} + Cl^{-}_{(aq)} -> NaCl_{(aq)}$ & $\Delta H_{NaCl} = +407,0 kJ \cdot mol^{-1}$ \\
                    $H^{+}_{(aq)} + (1/2) O_{2}_{(g)} \rightarrow H_{2}O_{(l)}$  & $\Delta H_{H2O} = +285,8 kJ \cdot mol^{-1}$\\
                    \hline
                    $NaOH_{(s)} + HCl_{(aq)} \rightarrow NaCl_{(aq)} + H_{2}O_{(l)}$ & $\Delta H_{neu} =  98,30 kJ \cdot mol^{-1}$ \\

				\end{tabular}\label{tab:table4}
            \end{table}
        
            \indent A equação da entalpia de dissolução de NaOH em HCl pode ser encontrada através das equações globais de neutralização já calculadas, através da lei de Hess, da seguinte forma:\\

            \begin{table}[h]
                \centering
                \renewcommand{\arraystretch}{2}
                \begin{tabular}{ll}
                    $NaOH_{(s)} + HCl_{(aq)} \rightarrow NaCl_{(aq)} + H_{2}O_{(l)}$ & $\Delta H_{neu} =  98,30 kJ \cdot mol^{-1}$ \\
                    $NaCl_{(aq)} + H_{2}O_{(l)} \rightarrow NaOH_{(aq)} + HCl_{(aq)}$ & $\Delta H_{neu} = -54,60 kJ \cdot mol^{-1}$ \\
                    \hline
                    $NaOH_{(s)} \rightarrow NaOH_{(aq)}$ & $\Delta H_{disol} = +43,7 \cdot kJ \cdot mol^{-1}$
                \end{tabular}
            \end{table}

        	\indent E a quantidade de calor liberado, segundo o modelo teórico, esperado para as condições em que foram desenvolvidos os experimentos, são:\\
        	\begin{equation}
        		\begin{split}
        			\notag
        			\Delta{H_{real_2}} &= \Delta{H_{neu}} \cdot C \cdot V_{solucao}\\
        			\Delta{H_{real_2}} &=(54,6 kJ \cdot mol^{-1}) (0,25 \cdot mol \cdot L^{-1})(0,1 \cdot L) \\
        			&= 1,365 \cdot kJ\\
        		\end{split}
        	\end{equation}\\
        	
			\indent Este experimento teve um preparo diferente do primeiro, uma vez que ao preparar a solução de $50\cdot mL$, partiu-se da solução de ácido clorídrico a $0.5\cdot mol / L$ que foi completada com água destilada até $100\cdot mL$, ou seja, $V = 0.1\cdot L$ a $C = 0.25 \cdot mol / L$. Os novos valores a serem listados, são:\
			
            \begin{enumerate}
                \item $T_{final} = 29,5 ^{\circ}C$
            	\item $C = 0,25 \cdot mol \cdot L^{-1}$
                \item $V = 0,1 \cdot L$
                \item $m_{NaOH} = 1,0 \cdot g$
                \item $m_{HCl} = 0,912 \cdot g$
                \item $m_{soluto} = 1,912 \cdot g$
                \item $m_{solucao} = 101,912 \cdot g$
            \end{enumerate}
        
        	\indent O procedimento de cálculo da parte experimental, do calor de dissolução nesta segunda etapa, tem como base o experimento 1, mantendo grande parte dos termos utilizados, sendo os novos valores, a nova temperatura final, a nova concentração da solução.\ Dessa forma, o cálculo do calor de dissolução experimental é:\

            \begin{equation}
            	\begin{split}
            		\notag
            		\Delta{H_{neu}} &= \Delta Q_{vidro} + \Delta Q_{solucao} \\
            		\Delta{H_{neu}} &= (m_{vidro} \cdot c_{vidro} \cdot \Delta T) + (m_{solucao} \cdot c_{solucao} \cdot \Delta T) \\
            		\Delta{H_{neu}} &= (m_{vidro} \cdot c_{vidro} + m_{solucao} \cdot c_{solucao}) \cdot \Delta T \\
            		\Delta{H_{neu}} &= (128,192 \cdot 0,67 + 101,912 \cdot 4,17) \cdot (29,5 - 23,4) \cdot J \\
            		\Delta{H_{neu}} &= 3,116 \cdot kJ
            	\end{split}
            \end{equation}\\
            
        
        \indent Em síntese, foi possível determinar as entalpias dos três processos químicos citados, sendo eles:
        
        \begin{enumerate}
        	\item A entalpia de neutralização de soluções de NaOH e HCl: $\Delta H_{1} = 54,60 \cdot kJ \cdot mol ^{-1}$
        	\item A entalpia de neutralização do HCl com NaOH sólido: $\Delta H_{2} = 98,30 \cdot kJ \cdot mol^{-1}$
        	\item A entalpia de solubilização do NaOH: $\Delta H_{3} = 43,70 \cdot kJ \cdot mol^{-1}$
        \end{enumerate}
            

        

