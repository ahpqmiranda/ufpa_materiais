%! Author = alanmiranda
%! Date = 14/11/2022
\section{Objetivos}\label{sec:objetivos}

\indent Como no exemplo citado na introdução, o fenômeno observado foi a "Dissolução". Trata-se do processo de dissolver um soluto em um solvente. Em tal procedimento, os solutos podem gerar íons,
que podem ser ou não classificados como eletrólitos.\ É dito eletrólito quando a dissolução acontece, ainda podem ser chamados de fortes, quando se dissociam completamente em um processo
irreversível.

\indent Os mais proeminentes exemplos de substâncias com forte caráter eletrólito são os sais, que são compostos formados por íons de cátions e ânions.\ Os cátions são íons positivos, e os ânions são íons negativos.\
Entre os eletrolíticos fracos, podemos citar os ácidos e as bases, que são compostos que se dissociam parcialmente em um processo reversível.\ Os ácidos são compostos que liberam íons H$^+$,
e as bases são compostos que liberam íons OH$^-$. Os eletrólitos fracos são regidos por uma constante de equilíbrio que nos permite determinar suas quantidades em função das condições impostas ao sistema químico.\\

\indent Dessa forma, o objetivo deste material passa por:

\begin{itemize}
        \item Verificar o aumento de temperatura de uma reação de neutralização, através de uma solução de NaOH, quando exposta a uma solução de HCl.
        \item Verificar o aumento de temperatura de um processo de dissolução, seguido de uma reação de neutralização de HCl, quando exposta ao NaOH em estado sólido, e comparar com o aumento de temperatura observado na mistura das soluções.
    \end{itemize}
    \subsection{Objetivos específicos}\label{sec:objetivos_especificos}
    \begin{itemize}
        \item Realizar a medição da variação de temperatura da reação de neutralização, e da dissolução, e comparar os resultados com os obtidos na literatura.
        \item Verificar a validade da Lei de Hess, que diz que a variação de energia livre de uma reação química é igual à soma das variações de energia livre das reações que a compõem.
        \item Trabalhar com a dissolução de ácidos e bases.
        \item Trabalhar conceitos de calorimetria dentro de um ambiente controlado.
        \item Calcular a variação de entalpia observada em um processo de dissolução, e de uma reação de neutralização.
    \end{itemize}