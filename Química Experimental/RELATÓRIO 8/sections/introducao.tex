    \section{Introdução}\label{sec:intro}
    \indent A termoquímica é o ramo da química que estuda as relações entre as energias envolvidas nas reações químicas e o seu efeito sobre a temperatura.\ A termoquímica pode ser usada para calcular quantidades de energia, como a quantidade de calor envolvida em uma reação, ou para determinar as condições ideais para uma reação ocorrer.

    \indent A primeira lei da termodinâmica é a lei da conservação da energia, que afirma que a energia total de um sistema é constante.\ A segunda lei da termodinâmica é a lei da entropia, que afirma que o aumento da entropia de um sistema é diretamente proporcional ao tempo.\ E a terceira lei, também conhecida como a lei da zero da termodinâmica, que afirma que a entropia de um sistema não pode ser negativa.\ Todas as leis da termodinâmica são aplicadas à química, e são usadas para prever o comportamento de reações químicas.

    \indent A quantidade de calor envolvida em uma reação química é medida pelo calor de reação, ou $q$.\ O calor de reação é a quantidade de calor liberado ou absorvido pelo sistema quando uma reação ocorre.\ O calor de reação pode ser calculado usando a equação $q = \Delta H$, onde $\Delta H$ é a mudança de entalpia da reação.\ A mudança de entalpia é a diferença entre a energia final do sistema e a energia inicial do sistema.

    \indent No exemplo abaixo, a reação é a decomposição do ácido sulfúrico em água e gás sulfúrico.

    \begin{center}
        \schemestart $2 H_2SO_4(aq)$ \arrow{--->} $2 H_2O(l) + SO_2(g)$ \schemestop \label{fig:exemplo}
    \end{center}

    \indent Neste exemplo, $q = \Delta H = -184 kJ$.\ Isto significa que 184 kJ de energia serão liberados quando 2 mol de ácido sulfúrico se decompuser em água e gás sulfúrico.



