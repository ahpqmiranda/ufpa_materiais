\documentclass{article}%
\usepackage[T1]{fontenc}%
\usepackage[utf8]{inputenc}%
\usepackage{lmodern}%
\usepackage{textcomp}%
\usepackage{lastpage}%
\usepackage{geometry}%
\geometry{tmargin=1.5cm,lmargin=1.5cm}%
%
%
%
\begin{document}%
\normalsize%
\section{RESENHA CRÍTICA: RECICLAGEM}%
\label{sec:RESENHACRTICARECICLAGEM}%
Visando a necessidade global da indústria em desenvolver meios, bem como buscar a viabilidade econômica na reciclagem de resíduos de componentes eletrônicos, o estudo busca dar finalidade a reciclagem de tubos de TV e monitores antigos, uma vez que são feitos com inúmeros materiais não degradáveis e alguns até tóxicos que são despejados na natureza, haja vista o descarte descontrolado nos países desenvolvidos que chegam na casa dos milhões a cada ano. %
Devido as questões apresentadas, foi realizado um trabalho de coleta e separação destes resíduos através de marca fabricante e modelo, bem como estudos a fim de analisar sua composição e testar as propriedades mecânicas, desta forma foi possível caracterizar os compostos, fabricar corpos de teste com os mesmos e descobrir que suas propriedades não se alteraram após vários anos expostos à natureza, bem como demonstraram propriedades mecânicas satisfatórias de forma a viabilizar a reciclagem, desta forma foi constatada a necessidade de elaboração de planos de coleta e tratamento dos resíduos para a reaplicação industrial.%
\subsection{RESENHA CRÍTICA: RECICLAGEM DE CARCAÇAS DE MONITORES; PROPRIEDADES MECÂNICAS E MORFOLÓGICAS}%
\label{subsec:RESENHACRTICARECICLAGEMDECARCAASDEMONITORESPROPRIEDADESMECNICASEMORFOLGICAS}%
O problema central observado é o descarte crescente do chamado lixo eletrônico, como pilhas, baterias, lâmpadas, celulares, computadores etc. por terem um período de vida útil curto e se tornarem obsoletos num prazo de 2{-}5 anos, e a escassez de locais de descartes, somados a obsolescência programada dos fabricantes agravam o problema, o Brasil como forma de combater à questão, elaborou a Política Nacional de Resíduos Sólidos (PNRS), Lei 12.305 em agosto de 2010 como forma de obrigar os fabricantes a desenvolver ações de coleta de seus produtos sucateados dos consumidores.%
A principal razão de coleta e reciclagem de componentes eletrônicos são as Placas de Circuito Integrado (PCI) na qual encontram{-}se metais raros, o que torna a reciclagem economicamente viável, a questão é a reciclagem de materiais polímeros como a Acrilonitrila Butadieno Estireno (ABS), poliestireno de alto impacto (HIPS) e policarbonato (PC) O Brasil é um grande usuário de reciclagem mecânica, processo que consiste em moagem, lavagem, secagem, aglutinação e o reprocessamento da matéria, a questão é que tal processo degrada os polímeros e os fazem perder propriedades mecânicas, porém após estudos, concluiu{-}se que mesmo com a degradação observada nos materiais, a aplicação comercial ainda seria válida.%
Houve a seleção dos materiais de acordo com ano de fabricação, marca, tamanho e cor do polímero, filtrando os detalhes mais técnicos do artigo em relação das condições de teste, os testes foram através de avaliação de propriedade física através de corpo de prova, onde foi realizado ensaios de impacto, tração, flexão e dureza de acordo com a norma ASTM D{-}256{-}06, em todos os casos, os ensaios foram de natureza destrutiva e foi realizada análises morfológicas das amostras nos locais de fratura.

%
\end{document}