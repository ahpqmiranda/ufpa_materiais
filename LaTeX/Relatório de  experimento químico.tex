\documentclass{article}%
\usepackage[T1]{fontenc}%
\usepackage[utf8]{inputenc}%
\usepackage{lmodern}%
\usepackage{textcomp}%
\usepackage{lastpage}%
\usepackage{longtable}%
\usepackage{tabu}%
\usepackage[table]{xcolor}%
\usepackage{booktabs}%
%
\usepackage[sorting=none]{biblatex}%
\addbibresource{biblatex{-}examples.bib}%
%
\begin{document}%
\normalsize%
\section{1. Introdução}%
\label{sec:1.Introduo}%

    Na introdução dará uma apresentação dos conceitos para o leitor que não conhece
    o assunto.       
    Sugestão: comece geral e depois vá ficando específico.       
    O que falar introdução? Suponha uma análise de acidez total do vinagre por       
    titulação volumétrica de neutralização. Comece geral e vá ficando específico. Comece       
    sobre química analítica, depois análise volumétrica, volumetria de neutralização, acidez       
    total. Fale do vinagre, o que é, como é produzido, legislação que regulamenta a acidez       
    aceitável em vinagres.
    

%
\section{2. Objetivos}%
\label{sec:2.Objetivos}%

        Podemos caracterizar como objetivos do experimento:
        \begin{itemize}
        \item Objetivo geral
            \begin{itemize}
            \Determinação de parâmetros de acidez, concentração, titulação, etc.
            \Dimensionamentos e Cálculos que podem vir a serem necessários.
            \end{itemize}
            
        \item Objetivos específicos
            \begin{itemize}
            \Preparo de Soluções.
            \Padronização de Soluções.
            \Diluições.
            \Planejamento de Experimentos.
            \end{itemize}
        \end{itemize}
        
        Há bancas de TCC que exigem ambos os objetivos.
        %
2.1 Objetivo geral%
2.2 Objetivos específicos

%
\section{3. Materiais e Métodos}%
\label{sec:3.MateriaiseMtodos}%

        Nesta secção, descreveremos os procedimentos adotados, métodos, equipamentos, reagentes, os preparos
        das soluções, se foram realizadas mudanças em relação a alguma eferência, etc.
        

%
\section{4. Resultados e Discussão}%
\label{sec:4.ResultadoseDiscusso}%

        Nesta secção, apresentamos os resultados obtidos e discutimos os mesmos.
        

%
\renewcommand{\arraystretch}{1.5}%
\begin{longtabu}{X[l] X[2l] X[r] X[r]}%
\rowcolor{lightgray}%
\textbf{nº}&\textbf{Elemento}&\textbf{Massa}&\textbf{Estado Físico}\\%
\hline%
1&Hidrogênio&1,00794&Gasoso\\%
\rowcolor{lightgray}%
2&Hélio&4,002602&Gasoso\\%
3&Lítio&6,941&Sólido\\%
\rowcolor{lightgray}%
4&Berílio&9,012182&Sólido\\%
5&Boro&10,811&Sólido\\%
5&elementos&massa&estado físico\\%
\rowcolor{lightgray}%
6&elementos&massa&estado físico\\%
7&elementos&massa&estado físico\\%
\rowcolor{lightgray}%
8&elementos&massa&estado físico\\%
9&elementos&massa&estado físico\\%
\rowcolor{lightgray}%
10&elementos&massa&estado físico\\%
11&elementos&massa&estado físico\\%
\rowcolor{lightgray}%
12&elementos&massa&estado físico\\%
13&elementos&massa&estado físico\\%
\rowcolor{lightgray}%
14&elementos&massa&estado físico\\%
15&elementos&massa&estado físico\\%
\rowcolor{lightgray}%
16&elementos&massa&estado físico\\%
17&elementos&massa&estado físico\\%
\rowcolor{lightgray}%
18&elementos&massa&estado físico\\%
19&elementos&massa&estado físico\\%
\rowcolor{lightgray}%
20&elementos&massa&estado físico\\%
21&elementos&massa&estado físico\\%
\rowcolor{lightgray}%
22&elementos&massa&estado físico\\%
23&elementos&massa&estado físico\\%
\rowcolor{lightgray}%
24&elementos&massa&estado físico\\%
\end{longtabu}%
\section{5. Conclusão}%
\label{sec:5.Concluso}%

        Aqui rebateremos os objetivos. Ex: Foi possível determinar a acidez da solução do vinagre
        O vinagre se mostrou dentro da faixa aceitável estipulada pelo mapa
        
        As comparações dos valores foram feitos nas discussões. Se tiver média e desvio padrão, pode calcular o
        intervalo de confiança (com um nível de confiança típico de 95%

%
\section{6. Referências}%
\label{sec:6.Referncias}%
\begin{tabular}{@{}X[l] X[2l] X[r] X[r]@{}}%
\toprule%
\midrule%
\textbf{Name}&\textbf{Desc}&\textbf{Number}&\textbf{Cite}\\%
\midrule%
foo&45&\Number{0}&\cite{askin}\\%
bar&23&\Number{1}&\cite{bar}\\%
Bar&7&\Number{2}&\cite{angenendt}\\%
Foobar&199&\Number{3}&\cite{doody}\\\bottomrule%
%
\end{tabular}

%
\end{document}