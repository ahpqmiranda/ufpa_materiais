\documentclass{article}%
\usepackage[T1]{fontenc}%
\usepackage[utf8]{inputenc}%
\usepackage{lmodern}%
\usepackage{textcomp}%
\usepackage{lastpage}%
\usepackage{geometry}%
\geometry{tmargin=1,0cm,lmargim=1.0cm,rmargim=1.0cm,bmargin=1.0cm}%
%
%
%
\begin{document}%
\normalsize%
\section{1. Introdução}%
\label{sec:1.Introduo}%

    Na introdução dará uma apresentação dos conceitos para o leitor que não conhece
    o assunto.       
    Sugestão: comece geral e depois vá ficando específico.       
    O que falar introdução? Suponha uma análise de acidez total do vinagre por       
    titulação volumétrica de neutralização. Comece geral e vá ficando específico. Comece       
    sobre química analítica, depois análise volumétrica, volumetria de neutralização, acidez       
    total. Fale do vinagre, o que é, como é produzido, legislação que regulamenta a acidez       
    aceitável em vinagres.
    

%
\section{2. Objetivos}%
\label{sec:2.Objetivos}%

        Podemos caracterizar como objetivos do experimento:
        \begin{itemize}
        \item Objetivo geral
            \begin{itemize}
            \Determinação de parâmetros de acidez, concentração, titulação, etc.
            \Dimensionamentos e Cálculos que podem vir a serem necessários.
            \end{itemize}
            
        \item Objetivos específicos
            \begin{itemize}
            \Preparo de Soluções.
            \Padronização de Soluções.
            \Diluições.
            \Planejamento de Experimentos.
            \end{itemize}
        \end{itemize}
        
        Há bancas de TCC que exigem ambos os objetivos.
        %
2.1 Objetivo geral%
2.2 Objetivos específicos

%
\section{3. Materiais e Métodos}%
\label{sec:3.MateriaiseMtodos}%

        Nesta secção, descreveremos os procedimentos adotados, métodos, equipamentos, reagentes, os preparos
        das soluções, se foram realizadas mudanças em relação a alguma eferência, etc.
        

%
\section{4. Resultados e Discussão}%
\label{sec:4.ResultadoseDiscusso}%

        Nesta secção, apresentamos os resultados obtidos e discutimos os mesmos.
        

%
\end{document}