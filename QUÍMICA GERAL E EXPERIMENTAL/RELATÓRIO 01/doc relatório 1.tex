%! Author = alana
%! Date = 09/09/2022

% Preamble
\documentclass[a4paper, 11pt]{article}

% Packages
\usepackage[brazil]{babel}
\usepackage[utf8]{inputenc}
\usepackage{amsmath, lmodern, amsthm, amstext, ebezier, amscd}
\usepackage{graphicx}
\usepackage[a4paper, left=2.5cm, right=2.5cm, top=2.5cm, bottom=2.5cm]{geometry}
\usepackage{setspace}
\usepackage{lipsum} \doublespacing
\usepackage{listings}
\usepackage{color}
\usepackage{indentfirst}
\usepackage{hyperref}


% Document
\begin{document}
\thispagestyle{empty} % Remove page number from first page to title page

\begin{center}
\parbox{3cm}{\includegraphics[scale=1]{logo_ufpa}} \\
{\vspace {1.0cm}}
{\Large \uppercase {Universidade Federal do Pará}}\\
{\Large \uppercase {Instituto de Ciências Exatas e Naturais - ICEN}}\\
\vspace{3cm}
{\Large \uppercase {Faculdade de Química - FAQUI}}\\
{\Large \uppercase {Laboratório de Química Analítica Quantitativa 2022.2} }\\
\vspace{3cm}
{\Large \textbf \uppercase {Relatório de Prática 1: Solução de Sulfato de Cobre II}}\\
{\Large \textbf \uppercase {Prof. Dr. Carlos Antonio Neves}}\\
\vspace{3cm}
{\Large \uppercase {Alan Henrique Pereira Miranda - 202102140072}}\\
{\Large \uppercase {Gabriel Cruz de Oliveira - 202102140055}}\\
{\Large \uppercase {Paloma Gama da Silva - 202102140029}}\\
{\Large \uppercase {Silvio Farias Leal - 202102140035}}\\
\vspace{0.5cm}
{\Large  {Belém \\ 2022}}
\end{center}

\newpage
\section{Introdução}\label{sec:introducao}

\indent As aulas de Química Experimental, permitem a oportunidade do aluno conhecer as diversas técnicas, procedimentos, instrumentos e atividades desenvolvidas por um químico em seu dia-a-dia.
Ao desenvolver um experimento químico, o aluno tem contato com uma variedade de equipamentos de laboratório, assim como suas finalidades específicas.
O emprego de um dado material ou equipamento depende de objetivos específicos e das condições em que serão realizados os experimentos.\\
\indent Este experimento tem por objetivo, ensinar e ambientar o aluno sobre conceitos, procedimentos laboratoriais e terminologia, bem como proporcionar o conhecimento de materiais e equipamentos básicos de um laboratório e suas aplicações.

\section{Objetivo}\label{sec:objetivo}
\indent O objetivo deste experimento é a produção e determinação da concentração de uma solução de sulfato de cobre II\@.
\begin{enumerate}

\item Conhecer as normas de segurança, bem como os equipamentos e materiais em um laboratório de Química.
\item Cuidados a serem tomados com os dentro do ambiente.
\item Conhecer os procedimentos básicos de preparação de soluções.

\end{enumerate}

\subsection*{Procedimento}

\end{document}